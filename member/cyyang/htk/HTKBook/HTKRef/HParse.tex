%/* ----------------------------------------------------------- */
%/*                                                             */
%/*                          ___                                */
%/*                       |_| | |_/   SPEECH                    */
%/*                       | | | | \   RECOGNITION               */
%/*                       =========   SOFTWARE                  */ 
%/*                                                             */
%/*                                                             */
%/* ----------------------------------------------------------- */
%/*         Copyright: Microsoft Corporation                    */
%/*          1995-2000 Redmond, Washington USA                  */
%/*                    https://www.microsoft.com                */
%/*                                                             */
%/*   Use of this software is governed by a License Agreement   */
%/*    ** See the file License for the Conditions of Use  **    */
%/*    **     This banner notice must not be removed      **    */
%/*                                                             */
%/* ----------------------------------------------------------- */
%
% HTKBook - Phil Woodland and Steve Young    24/11/97
%

\newpage
\mysect{HParse}{HParse}

\mysubsect{Function}{HParse-Function}

\index{hparse@\htool{HParse}|(}
The \htool{HParse} program generates word level lattice files (for use
with e.g. \htool{HVite}) from a text file syntax description containing a 
set of rewrite rules based on extended Backus-Naur Form (EBNF). 
The EBNF rules are used to generate an internal
representation of the corresponding finite-state network where \htool{HParse}
network nodes represent the words in the network, and are connected via
sets of links. This \htool{HParse} network is then converted to \HTK\ V2 word
level lattice. The program provides one convenient way of defining such
word level lattices. 

\htool{HParse} also provides a {\em compatibility mode} for use
with \htool{HParse} syntax descriptions used in \HTK\ V1.5 where
the same format was used to define both the word level syntax 
and the dictionary.
In compatibility mode \htool{HParse} will output the word level
portion of such a syntax as an \HTK\ V2 lattice file (via \htool{HNet})
and the pronuciation information as an \HTK\ V2 dictionary file (via 
\htool{HDict}).

The lattice produced by \htool{HParse} will often contain a number
of \texttt{!NULL} nodes in order to reduce the number of arcs in the
lattice. The use of such \texttt{!NULL} nodes can both
reduce size and  increase efficiency when used by recognition programs 
such as \htool{HVite}.

\mysubsect{Network Definition}{HParse-Network Definition}

The syntax rules for the textual definition of the network are 
as follows.  Each node in the network has a \texttt{nodename}.
This node name will normally correspond to a word in the final syntax
network. Additionally, for use in compatibility mode,
each node can also have an external name.
{\sf
\begin{tabbing}
++++++ \= ++++++++ \= ++ \= \kill
\>        name \> = \> char\{char\} \\
\>        nodename \> = \> name [ "\%" ( "\%" $|$ name ) ]
\end{tabbing}}

\noindent
Here \texttt{char} represents any character except one of the meta chars 
\texttt{\{ \} [ ] $<$ $>$$|$ = \$ ( ) ; $\backslash$ / *}.   The latter may, 
however, be escaped using a backslash.  The first name in a \texttt{nodename}
represents the name of the node (``internal name''), and the second optional name is
the ``external'' name.  This is used only in compatibility mode, and is, by default
the same as the internal name.

Network definitions may also contain variables
{\sf
\begin{tabbing}
++++++ \= ++++++++ \= ++ \= \kill
\>      variable \> = \> \$name
\end{tabbing}}
\noindent
Variables are identified by a leading \$ character.  They stand for
sub-networks and must be defined before they appear in the RHS of a rule
using the form
{\sf
\begin{tabbing}
++++++ \= ++++++++ \= ++ \= ++++++++++++ \=  \kill
\>      subnet \> = \> variable ``='' expr ``;''
\end{tabbing}}
\noindent
An \texttt{expr} consists of a set of alternative sequences representing
parallel branches of the network. 
{\sf
\begin{tabbing}
++++++ \= ++++++++ \= ++ \= ++++++++++++ \=  \kill
\>      expr \>  = \> sequence \{``$|$'' sequence\} \\
\>      sequence \> = \> factor\{factor\}
\end{tabbing}}
\noindent
Each sequence is composed of a sequence of factors where a factor
is either a node name, a variable representing some sub-network or
an expression contained within various sorts of brackets.

{\sf
\begin{tabbing}
++++++ \= ++++++++ \= ++ \= ++++++++++++ \=  \kill
\>   factor \> = \> ``('' expr ``)'' \> $|$ \\
\>\>\>            ``\{'' expr ``\}'' \> $|$ \\
\>\>\>            ``$<$'' expr ``$>$'' \> $|$ \\
\>\>\>         ``['' expr ``]'' \>  $|$ \\
\>\>\>             ``$<<$'' expr ``$>>$'' \> $|$ \\
\>\>\>               nodename \> $|$ \\
\>\>\>               variable 
\end{tabbing}}

Ordinary parentheses are used to bracket sub-expressions, curly braces \{ \} denote zero or more repetitions and angle brackets $<>$ denote one or more repetitions.  Square brackets [$\:$] are used to enclose optional items.  The double angle brackets are a special feature included for building context dependent loops and are explained further below.
Finally, the complete network is defined by a list of sub-network
definitions followed by a single expression within parentheses.
{\sf
\begin{tabbing}
++++++ \= ++++++++ \= ++ \= ++++++++++++ \=  \kill
\>    network \> = \> \{subnet\} ``('' expr ``)''
\end{tabbing}}
\noindent
Note that C style comments may be placed anywhere in the text of
the network definition.

As an example, the following network defines a syntax for some
simple edit commands
\begin{verbatim}
   $dir   = up | down | left | right;
   $mvcmd = move $dir | top | bottom;      
   $item  = char | word | line | page;
   $dlcmd = delete [$item];   /* default is char */
   $incmd = insert;
   $encmd = end [insert];
   $cmd = $mvcmd|$dlcmd|$incmd|$encmd;
   ({sil} < $cmd {sil} > quit)
\end{verbatim}

Double angle brackets are used to
construct contextually consistent context-dependent loops such
as a word-pair grammar.\footnote{The expression between 
double angle brackets must be a simple list of alternative node names or
a variable which has such a list as its value}
This function can also be used to generate consistent triphone loops 
for phone recognition\footnote{In \HTK\ V2 it is preferable for
these context-loop expansions to be done automatically via \htool{HNet},
to avoid requiring a dictionary entry for every context-dependent
model}.
The entry and exit conditions to a
context-dependent loop can be controlled by the invisible
pseudo-words TLOOP\_BEGIN and TLOOP\_END.  The right context of TLOOP\_BEGIN
defines the legal loop start nodes, and the left context of TLOOP\_END
defines the legal loop finishers. If TLOOP\_BEGIN/TLOOP\_END are not
present then all models are connected to the entry/exit of the loop.

A word-pair grammar simply defines the legal
set of words that can follow each word in the vocabulary.
To generate a network to represent such a grammar a
right context-dependent loop could be used.
The legal sentence set of sentence start and end words are defined as
above using TLOOP\_BEGIN/TLOOP\_END.

For example, the following lists the legal followers for each
word in a 7 word vocabulary

\begin{tabbing}
+++\=++++++++\=++\= \kill
\> ENTRY \> - \> show, tell, give \\
\> show \> - \> me, all \\
\> tell \> - \> me, all \\
\> me \> - \> all \\
\> all \> - \> names, addresses \\
\> names \> - \> and, names, addresses, show, tell, EXIT \\
\> addresses \> - \> and, names, addresses, show, tell, EXIT \\
\> and \> - \>  names, addresses, show, tell
\end{tabbing}

\htool{HParse} can generate a suitable lattice to represent this word-pair
grammar by using the following specification:
\begin{verbatim}
   $TLOOP_BEGIN_FLLWRS = show|tell|give;
   $TLOOP_END_PREDS    = names|addresses;
   $show_FLLWRS        = me|all;
   $tell_FLLWRS        = me|all;
   $me_FLLWRS          = all;
   $all_FLLWRS         = names|addresses;
   $names_FLLWRS       = and|names|addresses|show|tell|TLOOP_END;
   $addresses_FLLWRS   = and|names|addresses|show|tell|TLOOP_END;
   $and_FLLWRS         = names|addresses|show|tell;
     
   ( sil << 
         TLOOP_BEGIN+TLOOP_BEGIN_FLLWRS |
         TLOOP_END_PREDS-TLOOP_END |
         show+show_FLLWRS |
         tell+tell_FLLWRS |
         me+me_FLLWRS |
         all+all_FLLWRS |
         names+names_FLLWRS |
         addresses+addresses_FLLWRS |
         and+and_FLLWRS 
     >> sil )     
\end{verbatim}
where it is assumed that each utterance begins and ends with \texttt{sil}
model. 

In this example, each set of contexts is defined by creating a variable
whose alternatives are the individual contexts.  The actual context-dependent
 loop is indicated by the \texttt{<<  >>} brackets.  
Each element in this loop is a single
variable name of the form \texttt{A-B+C} where \texttt{A} represents the left
context, \texttt{C} represents the right context and \texttt{B} is the actual word.
Each of \texttt{A}, \texttt{B} and \texttt{C} can be nodenames or
variable names but note that this is the only case where variable
names are expanded automatically and the usual
\texttt{\$} symbol is not used\footnote{If the base-names or left/right context of the context-dependent names in a context-dependent loop are variables,  
no \texttt{\$} symbols are used when writing the context-dependent
nodename.}.  Both \texttt{A} and \texttt{C} are optional, and left and
right contexts can be mixed in the same triphone loop.

\mysubsect{Compatibility Mode}{HParse-Compatibility Mode}
In \htool{HParse} compatibility mode, the interpretation of the 
ENBF network is that used by the \HTK\ V1.5 \htool{HVite} program.
in which \htool{HParse} ENBF notation was used to define both the word 
level syntax and the dictionary.
Compatibility mode is aimed at converting files written for
\HTK\ V1.5 into their equivalent \HTK\ V2 representation.
Therefore \htool{HParse} will output the word level
portion of such a ENBF syntax as an \HTK\ V2 lattice file and the
pronunciation information is optionally stored in
an \HTK\ V2 dictionary file. When operating in compatibility mode
and not generating dictionary output, the pronunciation information
is discarded.

In compatibility mode, the reserved
node names \texttt{WD\_BEGIN} and \texttt{WD\_END} are used to delimit word 
boundaries---nodes between a \texttt{WD\_BEGIN/WD\_END} pair are called 
``word-internal'' while all other nodes are  ``word-external''.  
All \texttt{WD\_BEGIN/WD\_END} nodes 
must have an ``external name'' attached that denotes the word. 
It is a requirement that the number of \texttt{WD\_BEGIN} and the number
of \texttt{WD\_END} nodes are equal and furthermore that there isn't
a direct connection from a \texttt{WD\_BEGIN} node to a \texttt{WD\_END}.
For example a portion of such an \HTK\ V1.5  network could be 
\begin{verbatim}
     $A        =  WD_BEGIN%A ax WD_END%A;
     $ABDOMEN  =  WD_BEGIN%ABDOMEN ae b d ax m ax n WD_END%ABDOMEN;
     $ABIDES   =  WD_BEGIN%ABIDES ax b ay d z WD_END%ABIDES;
     $ABOLISH  =  WD_BEGIN%ABOLISH ax b aa l ih sh WD_END%ABOLISH;
      ... etc


     ( < 
        $A | $ABDOMEN | $ABIDES | $ABOLISH | ... etc
     > )
\end{verbatim}
\htool{HParse} will output the connectivity of the words 
in an HTK V2 word lattice format file
and the pronunciation information in an HTK V2 dictionary.
Word-external nodes are treated as words and stored in the lattice
with corresponding entries in the dictionary. 

It should be noted that in \HTK\ V1.5 any ENBF network could appear
between a \texttt{WD\_BEGIN/WD\_END} pair, which includes loops.
Care should therefore be taken with syntaxes that define very complex
sets of alternative pronunciations.  It should also be noted
that each dictionary entry is limited in length to 100 phones.
If multiple instances of the same word are found in the expanded
HParse network, a dictionary entry will be created for only the
first instance and subsequent instances are ignored (a warning is
printed). If words with a NULL external name are present then 
the dictionary will contain a NULL output symbol.

Finally, since the implementation of the generation of
the \htool{HParse} network has been revised\footnote{With the added benefit
of rectifying some residual bugs in the HTK V1.5 implementation}
the semantics of variable definition and use has been slightly changed. 
Previously variables could be redefined during network definition
and each use would follow the most recent definition. In HTK V2 only
the final definition of any variable is used in network expansion.

\mysubsect{Use}{HParse-Use}

\htool{HParse} is invoked via the command line
\begin{verbatim}
   HParse [options] syntaxFile latFile
\end{verbatim}
\htool{HParse} will then read the set of ENBF rules 
in \texttt{syntaxFile} and produce the output lattice in \texttt{latFile}.

The detailed operation of \htool{HParse} is controlled by the following
command line options
\begin{optlist}
  \ttitem{-b} Output the lattice in binary format. This increases
              speed of subsequent loading (default ASCII text lattices).

  \ttitem{-c} Set V1.5 compatibility mode. Compatibility mode can also
              be set by using the configuration variable V1COMPAT
              (default compatibility mode disabled).

  \ttitem{-d s} Output dictionary to file {\tt s}. This is only
                a valid option when operating in compatibility mode.
                If not set no dictionary will be produced.

  \ttitem{-l} Include language model log probabilities in the output
              These  log probabilities are calculated as 
              $-\log (\mbox{number of followers})$ for each network node.
\end{optlist}
\stdopts{HParse}

\mysubsect{Tracing}{HParse-Tracing}

\htool{HParse} supports the following trace options where each
trace flag is given using an octal base
\begin{optlist}
   \ttitem{0001} basic progress reporting.
   \ttitem{0002} show final HParse network (before conversion to a lattice)
   \ttitem{0004} print memory statistics after HParse lattice generation
   \ttitem{0010} show progress through glue node removal.
\end{optlist}
Trace flags are set using the \texttt{-T} option or the  \texttt{TRACE} 
configuration variable.
\index{hparse@\htool{HParse}|)}


%%% Local Variables: 
%%% mode: latex
%%% TeX-master: "../htkbook"
%%% End: 
