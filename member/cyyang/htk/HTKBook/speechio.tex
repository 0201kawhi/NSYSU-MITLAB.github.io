%/* ----------------------------------------------------------- */
%/*                                                             */
%/*                          ___                                */
%/*                       |_| | |_/   SPEECH                    */
%/*                       | | | | \   RECOGNITION               */
%/*                       =========   SOFTWARE                  */ 
%/*                                                             */
%/*                                                             */
%/* ----------------------------------------------------------- */
%/* developed at:                                               */
%/*                                                             */
%/*      Speech Vision and Robotics group                       */
%/*      Cambridge University Engineering Department            */
%/*      https://svr-www.eng.cam.ac.uk/                          */
%/*                                                             */
%/*      Entropic Cambridge Research Laboratory                 */
%/*      (now part of Microsoft)                                */
%/*                                                             */
%/* ----------------------------------------------------------- */
%/*         Copyright: Microsoft Corporation                    */
%/*          1995-2000 Redmond, Washington USA                  */
%/*                    https://www.microsoft.com                 */
%/*                                                             */
%/*              2001  Cambridge University                     */
%/*                    Engineering Department                   */
%/*                                                             */
%/*   Use of this software is governed by a License Agreement   */
%/*    ** See the file License for the Conditions of Use  **    */
%/*    **     This banner notice must not be removed      **    */
%/*                                                             */
%/* ----------------------------------------------------------- */
%
% HTKBook - Steve Young 1/12/97
%

\mychap{Speech Input/Output}{speechio}

Many tools need to input parameterised speech data and \HTK\ provides 
a number of different methods for doing this:
\begin{itemize}
\item input from a previously encoded speech parameter file
\item input from a waveform file which is encoded as part of the 
       input processing
\item input from an audio device which is encoded as part of the 
       input processing.
\end{itemize}
For input from a waveform file, a large number of different file formats
are supported, including all of the commonly used CD-ROM formats.
Input/output for parameter files is limited to the standard \HTK\ file format
and the new Entropic Esignal format.


\sidepic{Tool.spio}{60}{}
All \HTK\ speech input\index{speech input} is controlled by configuration
parameters which give details of what processing operations to apply to each
input speech file or audio source.  This chapter describes speech input/output
in \HTK.  The general mechanisms are explained and the various configuration
parameters are defined.  The facilities for signal pre-processing, linear
prediction-based processing, Fourier-based processing and vector quantisation
are presented and the supported file formats are given.  Also described are the
facilities for augmenting the basic speech parameters with energy measures,
delta coefficients and acceleration (delta-delta) coefficients and for
splitting each parameter vector into multiple data streams to form
\textit{observations}.  The chapter concludes with a brief description of the
tools \htool{HList} and \htool{HCopy} which are provided for viewing,
manipulating and encoding speech files.

\mysect{General Mechanism}{genio}

The facilities for speech input and output in \HTK\ are provided
by five distinct modules: \htool{HAudio}, \htool{HWave},
\htool{HParm}, \htool{HVQ} and \htool{HSigP}.  The interconnections
between these modules are shown in Fig.~\href{f:Spmods}.  
\index{speech input!general mechanism}

\sidefig{Spmods}{62}{Speech Input Subsystem}{2}{
Waveforms
are read from files using \htool{HWave}, or are input direct from
an audio device using \htool{HAudio}.  In a few rare cases, such as
in the display tool \htool{HSLab}, only the speech waveform is needed.
However, in most cases the waveform is wanted in parameterised form and
the required encoding is performed by \htool{HParm}
using the signal processing operations defined in 
\htool{HSigP}.  The parameter vectors are output by \htool{HParm}
in the form of observations which are the basic units of data processed
by the \HTK\ recognition and training tools.  An observation contains all
components of a raw parameter vector but it may be possibly split into
a number of independent parts.  Each such part is regarded by a \HTK\ tool
as a statistically independent data stream.  Also, an observation
may include VQ indices attached to each data stream.  Alternatively,
VQ indices can be read directly from a parameter file in which case the
observation will contain only VQ indices.
}

Usually a \HTK\ tool will require a number of speech data files to be
specified on the command line.  In the majority of cases, these
files will be required in parameterised form.  Thus, the following example
invokes the \HTK\ embedded training tool \htool{HERest}
to re-estimate a set of models using the speech data
files \texttt{s1}, \texttt{s2}, \texttt{s3}, \ldots .  These are
input via the library module \htool{HParm} and they
must be in exactly the form needed by the models.
\begin{verbatim}
    HERest ...  s1 s2 s3 s4 ...
\end{verbatim}

However, if the external form of the speech data files is not in the
required form, it will often be possible to convert them automatically during
the input process.
To do this, configuration parameter values are specified whose function 
is to define exactly
how the conversion should be done.  
The key idea is that there is a 
\textit{source parameter kind} and \textit{target parameter kind}.
The source refers to the natural form of the data in
the external medium and the target refers to the form of the
data that is required internally by the \HTK\ tool.
The principle function of the speech
input subsystem is to convert the source parameter kind into the 
required target parameter kind. \index{speech input!automatic conversion}

Parameter kinds consist of a base form to which one or more
qualifiers may be attached where each qualifier consists of
a single letter preceded by an underscore character.\index{qualifiers}
Some examples of parameter kinds are
\begin{varlist}
   \fwitem{2cm}{WAVEFORM} simple waveform
   \fwitem{2cm}{LPC} linear prediction coefficients
   \fwitem{2cm}{LPC\_D\_E} LPC with energy and delta coefficients
   \fwitem{2cm}{MFCC\_C}  compressed mel-cepstral coefficients
\end{varlist}
\index{speech input!target kind}

The required source and target parameter kinds are specified
using the configuration parameters \texttt{SOURCEKIND}
\index{sourcekind@\texttt{SOURCEKIND}} and 
\texttt{TARGETKIND}\index{targetkind@\texttt{TARGETKIND}}.
Thus, if the following configuration parameters were defined
\begin{verbatim}
    SOURCEKIND = WAVEFORM
    TARGETKIND = MFCC_E
\end{verbatim}
then the speech input subsystem would expect each input file to contain
a speech waveform and it would convert it to mel-frequency cepstral
coefficients with log energy appended.

The source need not be a waveform.  For example, the configuration
parameters
\begin{verbatim}
    SOURCEKIND = LPC
    TARGETKIND = LPREFC
\end{verbatim}
would be used to read in files containing linear prediction coefficients
and convert them to reflection coefficients.

For convenience, a special parameter kind called
\texttt{ANON}\index{anon@\texttt{ANON}} is provided.  When the source is
specified as \texttt{ANON} then the actual kind of the source is determined
from the input file.  When \texttt{ANON} is used in the target kind, then it is
assumed to be identical to the source.  For example, the effect of the
following configuration parameters
\begin{verbatim}
    SOURCEKIND = ANON
    TARGETKIND = ANON_D
\end{verbatim}
would simply be to add delta coefficients to whatever the source form
happened to be.  
The source and target parameter kinds default to \texttt{ANON}
to indicate that by default
no input conversions are performed.  Note, however, that where two or more
files are listed on the command line, the meaning of
\texttt{ANON} will not be re-interpreted from one file to the next.  Thus, it
is a general rule, that any tool reading multiple source speech files requires
that all the files have the same parameter kind.

The conversions applied by \HTK's input subsystem can be complex and may
not always behave exactly as expected.  There are two facilities that can 
be used to help check and debug the set-up of the speech i/o
configuration parameters.
Firstly, the tool \htool{HList} simply displays speech data by listing it
on the terminal.  However, since \htool{HList}  uses the speech 
input subsystem like
all \HTK\ tools, if a value for \texttt{TARGETKIND} is set, then 
it will display the target
form rather than the source form.  This is the simplest way to check the form of
the speech data that will actually be delivered to a \HTK\ tool. 
\htool{HList}  is described
in more detail in section~\ref{s:UseHList} below.

Secondly, trace output can be generated from the \htool{HParm} module
by setting the \texttt{TRACE} configuration file parameter.  This is a
bit-string in which individual bits cover different parts of the
conversion processing.  The details are given in the reference section.

To summarise,  speech input in \HTK\ is controlled by  configuration
parameters. The key parameters  are \texttt{SOURCEKIND} and {\tt
TARGETKIND} which specify the source and target parameter kinds. 
These determine the end-points of the required input conversion.
However,  to properly specify the detailed steps in between,  more
configuration parameters must be defined.
These are described in subsequent sections.

\mysect{Speech Signal Processing}{sigproc}

In this section, the basic mechanisms involved in transforming a
speech waveform into a sequence of parameter vectors will be
described.  Throughout this section, it is assumed that the
\texttt{SOURCEKIND} is \texttt{WAVEFORM} and that data is being read from
a HTK format file via \htool{HWave}.  Reading from different format
files is described below in section~\ref{s:waveform}.
Much of the
material in this section also applies to data read direct from an audio 
device, the
additional features needed to deal with this latter case are
described later in section~\ref{s:audioio}.
\vspace{0.2cm}
\index{speech input!blocking}

The overall process is illustrated in Fig.~\href{f:Blocking}
which shows the sampled waveform being converted into a 
sequence of parameter blocks.  In general, \HTK\ regards
both waveform files and parameter files as being just
sample sequences, the only difference being that in the former
case the samples are 2-byte integers and in the latter they
are multi-component vectors.  The sample rate of the input
waveform will normally be determined from the input file
itself.  However, it can be set explicitly using the
configuration parameter \texttt{SOURCERATE}.  The period
between each parameter vector determines the output sample
rate and it is set using the configuration parameter 
\texttt{TARGETRATE}.  The segment of waveform used to determine
each parameter vector is usually referred to as a window
and its size is set by the
configuration parameter \texttt{WINDOWSIZE}.  Notice that the
window size and frame rate are independent.  Normally,
the window size will be larger than the frame rate so that
successive windows overlap as illustrated in 
Fig.~\href{f:Blocking}.
\index{sourcerate@\texttt{SOURCERATE}}
\index{targetrate@\texttt{TARGETRATE}}
\index{windowsize@\texttt{WINDOWSIZE}}

For example, a waveform sampled
at 16kHz 
would be converted into 100 parameter vectors per
second using a 25 msec window by setting the following
configuration parameters.
\begin{verbatim}
    SOURCERATE = 625
    TARGETRATE = 100000
    WINDOWSIZE = 250000
\end{verbatim}
Remember that all durations are specified in 100 nsec units\footnote{
The somewhat bizarre choice of 100nsec units originated in Version 1 of
\HTK\ when times were represented by integers and this unit was the best
compromise between precision and range.  Times are now represented by
doubles and hence the constraints no longer apply.  However, the need for backwards
compatibility means that 100nsec units have been retained.  The names
\texttt{SOURCERATE} and \texttt{TARGETRATE} are also non-ideal, 
\texttt{SOURCEPERIOD} and \texttt{TARGETPERIOD} would be better. 
}.

\sidefig{Blocking}{50}{Speech Encoding Process}{2}{}
Independent of what parameter kind is required, there are some simple
pre-processing operations that can be applied prior to performing the actual
signal analysis.\index{speech input!pre-processing} 
Firstly, the DC mean can be removed from the source waveform by setting the
Boolean configuration parameter
\texttt{ZMEANSOURCE}\index{zmeansource@\texttt{ZMEANSOURCE}} to true 
(i.e.\ \texttt{T}). This is useful when\index{speech input!DC offset}
the original analogue-digital conversion has added a DC offset to the
signal. It is applied to each window individually so that it can be
used both when reading from a file and when using direct audio
input\footnote{ This method of applying a zero mean is different to
HTK Version 1.5 where the mean was calculated and subtracted from the
whole speech file in one operation. The configuration variable
\texttt{V1COMPAT} can be set to revert to this older behaviour.}.

Secondly,  it is common practice to pre-emphasise
the signal by applying the first order difference equation
\hequation{
   {s^{\prime}}_n  = s_n - k\,s_{n-1}
}{preemp}
to the samples\index{speech input!pre-emphasis}
$\{s_n, n=1,N \}$ in each window.  Here $k$ is the
pre-emphasis\index{pre-emphasis} coefficient which should be in the range
$0 \leq k < 1$.  It is specified using the configuration
parameter \texttt{PREEMCOEF}\index{preemcoef@\texttt{PREEMCOEF}}.
Finally,
it is usually beneficial to taper the
samples in each window so that discontinuities at the window
edges are attenuated.  This is done by setting the
Boolean configuration 
parameter \texttt{USEHAMMING}\index{usehamming@\texttt{USEHAMMING}}
to true.
This applies the following transformation to the samples
$\{s_n, n=1,N\}$ in the window
\hequation{
   {s^{\prime}}_n = \left\{ 0.54 - 0.46 \cos \left( \frac{2 \pi (n-1)}{N-1}
         \right) \right\} s_n
}{ham}
When both pre-emphasis and Hamming windowing are enabled,
pre-emphasis is performed 
first.\index{speech input!Hamming window function} \index{Hamming Window}

In practice, all three of the above are usually applied.
Hence, a configuration file will typically contain the 
following
\begin{verbatim}
    ZMEANSOURCE = T
    USEHAMMING = T
    PREEMCOEF = 0.97
\end{verbatim}
Certain types of artificially generated waveform data can cause numerical
overflows with some coding schemes. In such cases adding a small amount of
random noise to the waveform data solves the problem. The noise is added
to the samples using
\hequation{
   {s^{\prime}}_n = s_n + q RND()
}{dither}
where $RND()$ is a uniformly distributed random value over the interval
$[-1.0, +1.0)$ and $q$ is the scaling factor. The amount of noise added 
to the data ($q$) is set with the configuration parameter
\index{adddither@\texttt{ADDDITHER}}\texttt{ADDDITHER} (default value $0.0$). 
A positive value causes the noise signal added to be the same every time
(ensuring that the same file always gives exactly the same results). With a
negative value the noise is random and the same file may produce slightly
different results in different trials.

One problem that can arise when processing speech waveform files obtained from
external sources, such as databases on CD-ROM, is that the
byte-order\index{byte-order} may be different to that used by the machine on
which \HTK\ is running. To deal with this problem, \htool{HWave} can perform
automatic byte-swapping in order to preserve proper byte order. \HTK\ assumes
by default that speech waveform data is encoded as a sequence of 2-byte
integers as is the case for most current speech databases\footnote{Many of the
more recent speech databases use compression. In these cases, the data may be
regarded as being logically encoded as a sequence of 2-byte integers even if
the actual storage uses a variable length encoding scheme.}. 
If the source format is known, then \htool{HWave} will also make an assumption
about the byte order used to create speech files in that format. It then checks
the byte order of the machine that it is running on and automatically performs
byte-swapping if the order is different. For unknown formats, proper byte order
can be ensured by setting the configuration parameter
\texttt{BYTEORDER}\index{byteorder@\texttt{BYTEORDER}} to \texttt{VAX} if the
speech data was created on a little-endian machine such as a VAX or an IBM PC,
and to anything else (e.g. \texttt{NONVAX}) if the speech data was created on a
big-endian machine such as a SUN, HP or Macintosh machine. \index{speech
input!byte order} 

The reading/writing of \HTK\ format waveform files can be further controlled
via the configuration parameters \texttt{NATURALREADORDER} and 
\texttt{NATURALWRITEORDER}. The effect and default settings of these parameters
are described in section~\href{s:byteswap}.
\index{byte swapping}
Note that \texttt{BYTEORDER} should not be used when \texttt{NATURALREADORDER}
is set to true.  Finally, note that \HTK\ can also byte-swap parameterised
files in a similar way provided that only the byte-order of each 4 byte float
requires inversion.  

\mysect{Linear Prediction Analysis}{lpcanal}

In linear prediction (LP) \index{linear prediction} analysis, the 
vocal tract transfer function
is modelled by an all-pole filter\index{all-pole filter} with transfer function\footnote{
Note that some textbooks define the denominator of equation~\ref{e:allpole}
as $1 - \sum_{i=1}^p a_i z^{-i}$ so that the filter coefficients are the
negatives of those computed by \HTK.}
\hequation{
H(z) = \frac{1}{\sum_{i=0}^p a_i z^{-i}}
}{allpole}
where $p$ is the number of poles and $a_0 \equiv 1$.
The filter coefficients $\{a_i \}$ are chosen to minimise
the mean square filter prediction error summed over the analysis
window.  The \HTK\ module \htool{HSigP} uses the \textit{autocorrelation
method} to perform this optimisation as follows.

Given a window of speech samples $\{s_n, n=1,N \}$,
the first $p+1$ terms of the autocorrelation sequence are
calculated from
\hequation{
r_i = \sum_{j=1}^{N-i} s_j s_{j+i}
}{autoco}
where $i = 0,p$.
The filter coefficients are then computed recursively
using a set of auxiliary coefficients $\{k_i\}$ which can be
interpreted as the reflection coefficients of an equivalent
acoustic tube and the prediction error $E$ which is initially
equal to $r_0$.  Let $\{k_j^{(i-1)} \}$ and $\{a_j^{(i-1)} \}$
be the reflection and filter coefficients for a filter of order
$i-1$, then a filter of order $i$ can be calculated in three steps.
Firstly, a new set of reflection coefficients\index{reflection coefficients} are calculated.
\hequation{
 k_j^{(i)} = k_j^{(i-1)}
}{kupdate1}
for $j = 1,i-1$ and
\hequation{
 k_i^{(i)} =  \left\{ r_i + 
          \sum_{j=1}^{i-1} a_j^{(i-1)} r_{i-j} \right\} / E^{(i-1)}
}{kupdate2}
Secondly, the prediction energy is updated.
\hequation{
E^{(i)} = (1 -  k_i^{(i)}  k_i^{(i)} ) E^{(i-1)}
}{Eupdate}
Finally, new filter coefficients are computed
\hequation{
a_j^{(i)} = a_j^{(i-1)} - k_i^{(i)} a_{i-j}^{(i-1)}
}{aupdate1}
for $j = 1,i-1$ and
\hequation{
a_i^{(i)} = - k_i^{(i)} 
}{aupdate2}
This process is repeated from $i=1$ through to the required filter order
$i=p$.

To effect the above transformation, the target parameter kind must
be set to either \texttt{LPC}\index{lpc@\texttt{LPC}} to obtain the LP filter parameters $\{a_i\}$ or
\texttt{LPREFC}\index{lprefc@\texttt{LPREFC}} to obtain the reflection coefficients  $\{k_i \}$.  The
required filter order must also be set using the configuration
parameter \texttt{LPCORDER}\index{lpcorder@\texttt{LPCORDER}}.
Thus, for example, the following configuration
settings would produce a target parameterisation
consisting of 12 reflection coefficients per vector.
\begin{verbatim}
    TARGETKIND = LPREFC
    LPCORDER = 12
\end{verbatim}

An alternative LPC-based parameterisation is obtained by setting the
target kind to  \texttt{LPCEPSTRA}\index{lpcepstra@\texttt{LPCEPSTRA}} to generate linear prediction cepstra.  
The cepstrum of a signal is computed by taking a Fourier (or similar)
transform of the log spectrum.  In the case of linear 
prediction cepstra\index{linear prediction!cepstra}, the
required spectrum is the linear prediction spectrum which can be obtained
from the Fourier transform of the filter coefficients. However, it can be shown
that the required cepstra can be more efficiently computed using 
a simple recursion
\hequation{
   c_n = -a_n - \frac{1}{n} \sum_{i=1}^{n-1} (n-i) a_i c_{n-i}
}{lpcepstra}
The number of cepstra generated need not be the same as the number of
filter coefficients, hence it is set by a separate configuration 
parameter called \texttt{NUMCEPS}\index{numceps@\texttt{NUMCEPS}}.

The principal advantage of cepstral coefficients is that they are 
generally decorrelated and this allows diagonal covariances
to be used in the HMMs.  However, one minor problem with 
them is that the higher order cepstra are numerically quite small and 
this results in
a very wide range of variances when going from the low to high cepstral 
coefficients\index{cepstral coefficients!liftering}.
\HTK\ does not have a problem with this but for pragmatic reasons such as
displaying model parameters, flooring variances, etc., it is convenient to re-scale
the cepstral coefficients to have similar magnitudes.  This is done by
setting the configuration parameter \texttt{CEPLIFTER}\index{ceplifter@\texttt{CEPLIFTER}} to some value $L$ to
\textit{lifter} the cepstra according to the following formula
\hequation{
  {c^{\prime}}_n = \left( 1 + \frac{L}{2} sin \frac{\pi n}{L}
      \right) c_n
}{ceplifter}

As an example, the following configuration parameters would
use a 14'th order linear prediction analysis to
generate 12 liftered LP cepstra per target vector
\begin{verbatim}
    TARGETKIND = LPCEPSTRA
    LPCORDER = 14
    NUMCEPS = 12
    CEPLIFTER = 22
\end{verbatim}
These are typical of the values needed to generate a good front-end
parameterisation for a speech recogniser based on linear prediction.
\index{cepstral analysis!LPC based}\index{cepstral analysis!liftering coefficient}

Finally, note that the conversions supported by \HTK\ are not limited to
the case where the source is a waveform.  \HTK\ can convert any
LP-based parameter into any other LP-based parameter.

\mysect{Filterbank Analysis}{fbankanal}

The human ear resolves frequencies non-linearly across the audio spectrum and
empirical evidence suggests that designing a front-end to operate in a similar
non-linear manner improves recognition performance.  A popular alternative to
linear prediction based analysis is therefore filterbank analysis since this
provides a much more straightforward route to obtaining the desired non-linear
frequency resolution.  However, filterbank amplitudes are highly correlated and
hence, the use of a cepstral transformation in this case is virtually mandatory
if the data is to be used in a HMM based recogniser with diagonal covariances.
\index{cepstral analysis!filter bank} \index{speech input!filter bank}

\HTK\ provides a simple Fourier transform based filterbank designed to
give approximately equal resolution on a mel-scale.  Fig.~\href{f:melfbank}
illustrates the general form of this filterbank.    As can be seen,
the filters used are triangular and they are equally spaced along the mel-scale
which is defined by
\hequation{
   \mbox{Mel}(f) = 2595 \log_{10}(1 + \frac{f}{700})
}{melscale}
To implement this filterbank, the window of speech data is
transformed\index{mel scale} using a Fourier transform and the magnitude is
taken.  The magnitude coefficients are then \textit{binned} by correlating them
with each triangular filter.  Here binning means that each FFT magnitude
coefficient is multiplied by the corresponding filter gain and the results
accumulated.  Thus, each bin holds a weighted sum representing the spectral
magnitude in that filterbank channel.\index{binning} As an alternative, the
Boolean configuration parameter
\texttt{USEPOWER}\index{usepower@\texttt{USEPOWER}} can be set true to use the
power rather than the magnitude of the Fourier transform in the binning
process.  \index{cepstral analysis!power vs magnitude}

\centrefig{melfbank}{110}{Mel-Scale Filter Bank}

\index{speech input!bandpass filtering}
Normally the triangular filters are spread over the whole frequency range from
zero upto the Nyquist frequency.  However, band-limiting is often useful to
reject unwanted frequencies or avoid allocating filters to frequency regions in
which there is no useful signal energy.  For filterbank analysis only, lower
and upper frequency cut-offs can be set using the configuration parameters
\texttt{LOFREQ}\index{lofreq@\texttt{LOFREQ}} and
\texttt{HIFREQ}\index{hifreq@\texttt{HIFREQ}}. For example,
\begin{verbatim}
    LOFREQ = 300
    HIFREQ = 3400
\end{verbatim}
might be used for processing telephone speech.  When low and high pass cut-offs
are set in this way, the specified number of filterbank channels are distributed
equally on the mel-scale across the resulting pass-band such that the lower cut-off
of the first filter is at \texttt{LOFREQ} and the upper cut-off of the last
filter is at \texttt{HIFREQ}.

If mel-scale filterbank parameters are required directly, then the target kind
should be set to \texttt{MELSPEC}\index{melspec@\texttt{MELSPEC}}.
Alternatively, log filterbank parameters can be generated by setting the target
kind to \texttt{FBANK}.  


\mysect{Vocal Tract Length Normalisation}{vtln}

A simple speaker normalisation technique can be implemented by
modifying the filterbank analysis described in the previous section.
Vocal tract length normalisation (VTLN) aims to compensate for the
fact that speakers have vocal tracts of different sizes. VTLN can be
implemented by warping the frequency axis in the filterbank analysis.
In HTK simple linear frequency warping is supported. The warping
factor~$\alpha$ is controlled by the configuration variable
\texttt{WARPFREQ}\index{melspec@\texttt{WARPFREQ}}. Here values of
$\alpha < 1.0$ correspond to a compression of the frequency axis. As
the warping would lead to some filters being placed outside the
analysis frequency range, the simple linear warping function is
modified at the upper and lower boundaries. The result is that the
lower boundary frequency of the analysis
(\texttt{LOFREQ}\index{melspec@\texttt{LOFREQ}}) and the upper
boundary frequency (\texttt{HIFREQ}\index{melspec@\texttt{HIFREQ}})
are always mapped to themselves. The regions in which the warping
function deviates from the linear warping with factor~$\alpha$ are
controlled with the two configuration variables
(\texttt{WARPLCUTOFF}\index{melspec@\texttt{WARPLCUTOFF}}) and
(\texttt{WARPUCUTOFF}\index{melspec@\texttt{WARPUCUTOFF}}).
Figure~\href{f:vtlnpiecewise} shows the overall shape of the resulting
piece-wise linear warping functions.

\centrefig{vtlnpiecewise}{60}{Frequency Warping}

The warping factor~$\alpha$ can for example be found using a search
procedure that compares likelihoods at different warping factors. A
typical procedure would involve recognising an utterance with
$\alpha=1.0$ and then performing forced alignment of the hypothesis
for all warping factors in the range $0.8 - 1.2$. The factor that
gives the highest likelihood is selected as the final warping factor.
Instead of estimating a separate warping factor for each utterance,
large units can be used by for example estimating only one~$\alpha$
per speaker. 

Vocal tract length normalisation can be applied in testing as well as
in training the acoustic models.

\mysect{Cepstral Features}{cepstrum}

Most often, however, cepstral parameters are required
and these are indicated by setting the target kind to \texttt{MFCC} standing
for Mel-Frequency Cepstral Coefficients (MFCCs).  These are calculated from the
log filterbank amplitudes $\{m_j\}$ using the Discrete Cosine Transform
\hequation{
   c_i = \sqrt{\frac{2}{N}} \sum_{j=1}^N m_j \cos \left( \frac{\pi i}{N}(j-0.5) \right)
}{dct}
where $N$ is the number of filterbank channels set by the configuration
parameter \texttt{NUMCHANS}\index{numchans@\texttt{NUMCHANS}}.  The required
number of cepstral coefficients is set by
\texttt{NUMCEPS}\index{numceps@\texttt{NUMCEPS}} as in the linear prediction
case.  Liftering can also be applied to MFCCs using the
\texttt{CEPLIFTER}\index{ceplifter@\texttt{CEPLIFTER}} configuration parameter
(see equation~\ref{e:ceplifter}).

MFCCs are the parameterisation of choice for many speech recognition applications.
They give good discrimination and lend themselves to a number of manipulations.
In particular, the effect of inserting a transmission channel on the input
speech is to multiply the speech spectrum by the channel transfer function.
In the log cepstral domain, this multiplication becomes a simple addition which
can be removed by subtracting the cepstral mean from all input vectors.
In practice, of course, the mean has to be estimated over a limited amount
of speech data so the subtraction will not be perfect.  Nevertheless, this
simple technique is very effective in practice where it
compensates for long-term spectral effects such as those caused by different
microphones and audio channels.  To perform this
so-called \textit{Cepstral Mean Normalisation} (CMN) in \HTK\, it is only necessary
to add the \texttt{\_Z}\index{qualifiers!aaaz@\texttt{\_Z}} qualifier to the 
target parameter kind.  The mean is estimated by computing the average of
each cepstral parameter across each input speech file.  Since this cannot be done
with live audio, cepstral mean compensation is not supported for this case.
\index{cepstral mean normalisation}

In addition to the mean normalisation the variance of the data can be
normalised. For improved robustness both mean and variance of the data
should be calculated on a larger units (e.g.\ on all the data from a
speaker instead of just on a single utterance). To use
speaker-/cluster-based normalisation the mean and variance estimates
are computed offline before the actual recognition and stored in
separate files (two files per cluster). The configuration variables
\texttt{CMEANDIR}\index{numchans@\texttt{CMEANDIR}} and
\texttt{VARSCALEDIR}\index{numchans@\texttt{VARSCALEDIR}} point to the
directories where these files are stored. To find the actual filename
a second set of variables
(\texttt{CMEANMASK}\index{numchans@\texttt{CMEANMASK}} and
\texttt{VARSCALEMASK}\index{numchans@\texttt{VARSCALEMASK}}) has to be
specified. These masks are regular expressions in which you can use
the special characters \texttt{?}, \texttt{*} and \texttt{\%}. The
appropriate mask is matched against the filename of the file to be
recognised and the substring that was matched against the \texttt{\%}
characters is used as the filename of the normalisation file. An
example config setting is:

\begin{verbatim}
CMEANDIR     = /data/eval01/plp/cmn
CMEANMASK    = %%%%%%%%%%_*
VARSCALEDIR  = /data/eval01/plp/cvn
VARSCALEMASK = %%%%%%%%%%_*
VARSCALEFN   = /data/eval01/plp/globvar
\end{verbatim}

So, if the file \verb|sw1-4930-B_4930Bx-sw1_000126_000439.plp| is to be
recognised then the normalisation estimates would be loaded from the
following files:

\begin{verbatim}
/data/eval01/plp/cmn/sw1-4930-B
/data/eval01/plp/cvn/sw1-4930-B
\end{verbatim}

The file specified by
\texttt{VARSCALEFN}\index{numchans@\texttt{VARSCALEFN}} contains the
global target variance vector, i.e. the variance of the data is first
normalised to 1.0 based on the estimate in the appropriate file in
\texttt{VARSCALEDIR}\index{numchans@\texttt{VARSCALEDIR}} and then
scaled to the target variance given in
\texttt{VARSCALEFN}\index{numchans@\texttt{VARSCALEFN}}.

The format of the files is very simple and each of them just contains
one vector. Note that in the case of the cepstral mean only the static
coefficients will be normalised. A cmn file could for example look like:

\begin{verbatim}
<CEPSNORM> <PLP_0>
<MEAN> 13
-10.285290 -9.484871 -6.454639 ...
\end{verbatim}


The cepstral variance normalised always applies to the full
observation vector after all qualifiers like delta and acceleration
coefficients have been added, e.g.:

\begin{verbatim}
<CEPSNORM> <PLP_D_A_Z_0>
<VARIANCE> 39
33.543018 31.241779 36.076199 ...
\end{verbatim}

The global variance vector will always have the same number of
dimensions as the cvn vector, e.g.:

\begin{verbatim}
<VARSCALE> 39
 2.974308e+01 4.143743e+01 3.819999e+01 ...
\end{verbatim}

These estimates can be generated using \htool{HCompV}. See the
reference section for details.

 
\mysect{Perceptual Linear Prediction}{plp}

An alternative to the Mel-Frequency Cepstral Coefficients is the use
of Perceptual Linear Prediction (PLP) coefficients.

As implemented in HTK the PLP feature extraction is based on the
standard mel-frequency filterbank (possibly warped). The mel
filterbank coefficients are weighted by an equal-loudness curve and
then compressed by taking the cubic root.\footnote{the degree of
  compression can be controlled by setting the configuration parameter
  \texttt{COMPRESSFACT}\index{enormalise@\texttt{COMPRESSFACT}} which
  is the power to which the amplitudes are raised and defaults to
  0.33)} From the resulting auditory spectrum LP coefficents are
estimated which are then converted to cepstral coefficents in the
normal way (see above).


\mysect{Energy Measures}{energy}

\index{speech input!energy measures} 
To augment the spectral parameters derived from linear prediction or
mel-filterbank analysis, an energy term can be appended by including the
qualifier \texttt{\_E}\index{qualifiers!aaae@\texttt{\_E}} in the target kind.
The energy is computed as the log of the signal energy, that is, for speech
samples $\{s_n, n=1,N \}$
\hequation{
   E = log \sum_{n=1}^N s_n^2
}{logenergy}

This log energy measure can be normalised to the range $-E_{min}..1.0$ by
setting the Boolean configuration parameter
\texttt{ENORMALISE}\index{enormalise@\texttt{ENORMALISE}} to true (default
setting).  This
normalisation is implemented by subtracting the maximum value of $E$ in the
utterance and adding $1.0$. 
Note that energy normalisation is incompatible with live audio 
input and in such circumstances the configuration variable \texttt{ENORMALISE} 
should be explicitly set false.
The lowest energy in the utterance can be clamped using the configuration
parameter
\texttt{SILFLOOR}\index{silfloor@\texttt{SILFLOOR}} which gives the ratio
between the maximum and minimum energies in the utterance in dB. Its default
value is 50dB. 
Finally, the overall log energy can be arbitrarily scaled by the value of the 
configuration parameter \texttt{ESCALE}\index{escale@\texttt{ESCALE}} whose 
default is $0.1$. \index{silence floor}

When calculating energy for LPC-derived parameterisations, the default is to
use the zero-th delay autocorrelation coefficient ($r_0$).  However, this means
that the energy is calculated after windowing and pre-emphasis.  If the
configuration parameter \texttt{RAWENERGY}\index{rawenergy@\texttt{RAWENERGY}}
is set true, however, then energy is calculated separately before any windowing
or pre-emphasis regardless of the requested parameterisation\footnote{ In any
event, setting the compatibility variable \texttt{V1COMPAT} to true in
\htool{HPARM} will ensure that the calculation of energy is compatible with
that computed by the Version 1 tool \htool{HCode}.  }.

In addition to, or in place of, the log energy, the qualifier
\texttt{\_O}\index{qualifiers!aaao@\texttt{\_O}} can be added to a target kind
to indicate that the 0'th cepstral parameter $C_0$ is to be appended.  This
qualifier is only valid if the target kind is \texttt{MFCC}.  Unlike earlier
versions of \HTK\, scaling factors set by the configuration variable
\texttt{ESCALE} are not applied to $C_0$\footnote{ Unless \texttt{V1COMPAT} is
set to true.  }.

\mysect{Delta, Acceleration and Third Differential Coefficients}{delta}

\index{speech input!dynamic coefficents}
The performance of a speech recognition system can be greatly enhanced by
adding time derivatives to the basic static parameters.  In \HTK, these are
indicated by attaching qualifiers to the basic parameter kind.  The qualifier
\texttt{\_D} indicates that first order regression coefficients (referred to as
delta coefficients) are appended, the qualifier
\texttt{\_A}\index{qualifiers!aaaa@\texttt{\_A}} indicates that second order
regression coefficients (referred to as acceleration coefficients) and 
 the qualifier
\texttt{\_T}\index{qualifiers!aaaa@\texttt{\_T}} indicates that third order
regression coefficients (referred to as third differential coefficients) are
appended. The \texttt{\_A} qualifier cannot be used without also using the
\texttt{\_D}\index{qualifiers!aaad@\texttt{\_D}} qualifier. Similarly
the \texttt{\_T} qualifier cannot be used without also using the
\texttt{\_D} and \texttt{\_A} qualifiers.

The delta coefficients\index{delta coefficients} are computed using the
following regression formula\index{regression formula}
\hequation{
   d_t = \frac{ \sum_{\theta =1}^\Theta \theta(c_{t+\theta} - c_{t-\theta}) }{
                2 \sum_{\theta = 1}^\Theta \theta^2 }
}{deltas}
where $d_t$ is a delta coefficient at time $t$ computed in terms of the
corresponding static coefficients $c_{t-\Theta}$ to $c_{t+\Theta}$.  The value
of $\Theta$ is set using the configuration parameter
\texttt{DELTAWINDOW}\index{deltawindow@\texttt{DELTAWINDOW}}.  The same formula
is applied to the delta coefficients to obtain acceleration coefficients except
that in this case the window size is set by
\texttt{ACCWINDOW}\index{accwindow@\texttt{ACCWINDOW}}. Similarly
the third differentials use \texttt{THIRDWINDOW}. Since
equation~\ref{e:deltas} relies on past and future speech parameter values, 
some modification is needed at the beginning and end of the speech.  The
default behaviour is to replicate the first or last vector as needed to fill
the regression window.

In older version 1.5 of \HTK\ and earlier, this end-effect problem was solved
by using simple 
first order differences at the start and end of the speech, that is
\begin{equation}
   d_t = c_{t+1} - c_t,\;\;\; t<\Theta
\end{equation}
and
\begin{equation}
   d_t = c_t - c_{t-1}, \;\;\; t \geq T-\Theta
\end{equation}
where $T$ is the length of the data file.  If required, this older behaviour
can be restored by setting the configuration variable 
\texttt{V1COMPAT}\index{v1compat@\texttt{V1COMPAT}}
to true in \htool{HParm}.

For some purposes, it is useful to use simple differences throughout.  This
can be achieved by setting the configuration 
variable \texttt{SIMPLEDIFFS}\index{simplediffs@\texttt{SIMPLEDIFFS}}
to true in \htool{HParm}.  In this case, just the end-points of the delta window
are used, i.e.
\hequation{
   d_t = \frac{ (c_{t+\Theta} - c_{t-\Theta}) }{
                2 \Theta}
}{simdiffs}
\index{simple differences}

When delta and acceleration coefficients are requested, they are computed for
all static parameters including energy if present.  In some applications, the
absolute energy is not useful but time derivatives of the energy may be.  By
including the \texttt{\_E} qualifier together with the
\texttt{\_N}\index{qualifiers!aaan@\texttt{\_N}} qualifier, the absolute energy
is suppressed leaving just the delta and acceleration coefficients of the
energy.

\mysect{Storage of Parameter Files}{parmstore}

Whereas \HTK\ can handle waveform data in a variety of file formats,
all parameterised speech data is stored externally in either native
\HTK\ format data files or Entropic Esignal format files.
Entropic ESPS format is no longer supported directly, but input and output
filters can be used to convert ESPS to Esignal format on input and
Esignal to ESPS on output.

\subsection{\HTK\ Format Parameter Files}

\HTK\ format files consist of a contiguous sequence of \textit{samples}
preceded by a header.  Each sample is a vector of either 2-byte integers or
4-byte floats.  2-byte integers are used for compressed forms as described
below and for vector quantised data as described later in
section~\ref{s:vquant}.  \HTK\ format data files can also be used to store
speech waveforms as described in section~\ref{s:waveform}.  \index{file
formats!HTK}

The \HTK\ file format header is 12 bytes long and contains the following data 
\begin{tabbing}
++ \= +++++++++ \= \kill
\>\texttt{nSamples}\>-- number of samples in file (4-byte integer) \\
\>\texttt{sampPeriod}\>-- sample period in 100ns units (4-byte integer) \\
\>\texttt{sampSize}\>-- number of bytes per sample (2-byte integer) \\
\>\texttt{parmKind}\>-- a code indicating the sample kind (2-byte integer)
\end{tabbing}
The parameter kind\index{parameter kind} consists of a 6 bit
code representing the basic parameter kind plus additional bits for
each of the possible qualifiers\index{qualifiers}.  The basic parameter kind codes are
\begin{tabbing}
++++\= +++ \= ++++++++ \=   \kill
\>0 \> \texttt{WAVEFORM} \> sampled waveform \\
\>1 \> \texttt{LPC} \> linear prediction filter coefficients \\
\>2 \> \texttt{LPREFC} \> linear prediction reflection coefficients \\
\>3 \> \texttt{LPCEPSTRA} \> LPC cepstral coefficients \\
\>4 \> \texttt{LPDELCEP}  \> LPC cepstra plus delta coefficients \\
\>5 \> \texttt{IREFC}       \> LPC reflection coef in 16 bit integer format  \\
\>6 \> \texttt{MFCC}      \> mel-frequency cepstral coefficients \\
\>7 \> \texttt{FBANK}     \> log mel-filter bank channel outputs \\
\>8 \> \texttt{MELSPEC}     \> linear mel-filter bank channel outputs \\
\>9 \> \texttt{USER}      \> user defined sample kind \\
\>10 \> \texttt{DISCRETE} \> vector quantised data \\
\>11 \> \texttt{PLP}     \> PLP cepstral coefficients \\
\end{tabbing}
and the bit-encoding for the qualifiers (in octal) is 
\begin{tabbing}
++++\= +++ \= ++++++++ \=   \kill
\>\texttt{\_E} \> 000100 \> has energy \\
\>\texttt{\_N} \> 000200 \> absolute energy suppressed \\
\>\texttt{\_D} \> 000400 \> has delta coefficients \\
\>\texttt{\_A} \> 001000 \> has acceleration coefficients\\
\>\texttt{\_C} \> 002000 \> is compressed \\
\>\texttt{\_Z} \> 004000 \> has zero mean static coef. \\
\>\texttt{\_K} \> 010000 \> has CRC checksum \\
\>\texttt{\_O} \> 020000 \> has 0'th cepstral coef. \\
\>\texttt{\_V} \> 040000 \> has VQ data  \\
\>\texttt{\_T} \> 100000 \> has third differential coef. \\
\end{tabbing}\index{qualifiers!codes}
The \texttt{\_A} qualifier can only be specified when \texttt{\_D}
is also specified.
The \texttt{\_N} qualifier is only valid when both energy and delta
coefficients are present.  
The sample kind \texttt{LPDELCEP} is identical to \texttt{LPCEPSTRA\_D}
and is retained for compatibility with older versions of \HTK.
The \texttt{\_C}\index{qualifiers!aaac@\texttt{\_C}} and 
\texttt{\_K}\index{qualifiers!aaak@\texttt{\_K}} only exist in external files.  Compressed
files are always decompressed on loading and any attached CRC 
is checked and removed.  An external file can contain both an energy
term and a 0'th order cepstral coefficient.  These may be retained
on loading but normally one or the other is discarded\footnote{
Some applications may require the 0'th order cepstral coefficient
in order to recover the filterbank coefficients from the cepstral
coefficients.}.

\putfig{HTKFormat}{130}{Parameter Vector Layout in \HTK\ Format Files}

All parameterised forms of \HTK\ data files consist of a sequence of vectors.
Each vector is organised as shown by the examples in Fig~\href{f:HTKFormat}
where various different qualified forms are listed.  As can be seen, an energy
value if present immediately follows the base coefficients.  If delta
coefficients are added, these follow the base coefficients and energy value.
Note that the base form \texttt{LPC} is used in this figure only as an example,
the same layout applies to all base sample kinds.  If the 0'th order cepstral
coefficient is included as well as energy then it is inserted immediately
before the energy coefficient, otherwise it replaces it.

For external storage of speech parameter files, two compression methods are
provided.  For LP coding only, the \texttt{IREFC} parameter kind exploits the
fact that the reflection coefficients are bounded by $\pm 1$ and hence they can
be stored as scaled integers such that $+1.0$ is stored as $32767$ and $-1.0$
is stored as $-32767$.  For other types of parameterisation, a more general
compression facility indicated by the
\texttt{\_C}\index{qualifiers!aaac@\texttt{\_C}} qualifier is used.  
\HTK\ compressed parameter files consist of a set of compressed parameter 
vectors stored as shorts such that for parameter $x$
\begin{eqnarray}
x_{short} & = & A*x_{float}-B  \nonumber   
\end{eqnarray}
The coefficients $A$ and $B$ are defined as
\begin{eqnarray}
A & = & 2*I/(x_{max}-x_{min}) \nonumber\\
B & = & (x_{max}+x_{min})*I/(x_{max}-x_{min}) \nonumber
\end{eqnarray}
where $x_{max}$ is the maximum value of parameter $x$ in the whole file and
$x_{min}$ is the corresponding minimum. $I$ is the maximum range of a 2-byte
integer i.e.\ 32767.  The values of $A$ and $B$ are stored as two floating
point vectors prepended to the start of the file immediately after the header.

When a \HTK\ tool writes out a speech file to external storage, no further
signal conversions are performed.  Thus, for most purposes, the target
parameter kind specifies both the required internal representation and the form
of the written output, if any.  However, there is a distinction in the way that
the external data is actually stored.  Firstly, it can be compressed as
described above by setting the configuration parameter \texttt{SAVECOMPRESSED}
to true.  If the target kind is \texttt{LPREFC} then this compression is
implemented by converting to \texttt{IREFC} otherwise the general compression
algorithm described above is used.  Secondly, in order to avoid data corruption
problems, externally stored \HTK\ parameter files can have a cyclic redundancy
checksum appended.  This is indicated by the qualifier
\texttt{\_K}\index{qualifiers!aaak@\texttt{\_K}} and it is generated by setting
the configuration parameter \texttt{SAVEWITHCRC} to true.  The principle tool
which uses these output conversions is \htool{HCopy} (see
section~\ref{s:UseHCopy}).

\subsection{Esignal Format Parameter Files}

\index{file formats!Esignal}
The default for parameter files is native \HTK\ format.  However, \HTK\ tools
also support the Entropic Esignal format for both input and output. Esignal
replaces the Entropic ESPS file format. To ensure compatibility Entropic
provides conversion programs from ESPS to ESIG and vice versa.

To indicate that a source file is in Esignal format the configuration
variable  \texttt{SOURCEFORMAT}\index{sourceformat@\texttt{SOURCEFORMAT}}
should be set to \texttt{ESIG}.  Alternatively, 
\texttt{-F ESIG}\index{standard options!aaaf@\texttt{-F}} can be specified
as a command-line option.  
To generate Esignal format output files, the configuration variable
\texttt{TARGETFORMAT} should be set to \texttt{ESIG} or the command line option
\texttt{-O ESIG} should be set.

ESIG files consist of three parts: a preamble, a sequence of field 
specifications called the field list and a sequence of records. The preamble
and the field list together constitute the header. The preamble is purely 
ASCII. Currently it consists of 6 information items that are all terminated
by a new line.  The information in the preamble is the following:
\begin{tabbing}
++ \= +++++++++ \= \kill
\>\texttt{line 1}\>-- identification of the file format \\
\>\texttt{line 2}\>-- version of the file format\\
\>\texttt{line 3}\>-- architecture (ASCII, EDR1, EDR2, machine name)\\
\>\texttt{line 4}\>-- preamble size (48 bytes)\\
\>\texttt{line 5}\>-- total header size\\
\>\texttt{line 6}\>-- record size\\
\end{tabbing}
All ESIG files that are output by \HTK\ programs contain the following 
global fields:
\begin{description}
  \item[commandLine] the command-line used to generate the file;
  \item[recordFreq] a double value that indicates the sample frequency
        in Herz;
  \item[startTime] a double value that indicates a time at which the first 
        sample is presumed to be starting;
  \item[parmKind] a character string that indicates the full 
        type of parameters in the file, e.g: \texttt{MFCC\_E\_D}.
  \item[source\_1] if the input file was an ESIG file this field includes the
        header items in the input file.
\end{description}
After that there are field specifiers for the records. The first specifier 
is for the  basekind of the parameters, e.g: \texttt{MFCC}. Then for each 
available qualifier there are additional specifiers. Possible specifiers are:
\begin{tabbing}
++++\=   \kill
\>\texttt{zeroc}  \\
\>\texttt{energy}\\
\>\texttt{delta}\\
\>\texttt{delta\_zeroc} \\
\>\texttt{delta\_energy}\\
\>\texttt{accs}\\
\>\texttt{accs\_zeroc} \\
\>\texttt{accs\_energy}\\
\end{tabbing}\index{qualifiers!ESIG field specifiers}
The data segments of the ESIG files have exactly the same format as the
the corresponding \HTK\ files. This format was described in the previous 
section.

\HTK\ can only input parameter files that have a valid parameter kind as value
of the header field \texttt{parmKind}. If this field does not exist or if the
value of this field does not contain a valid parameter kind, the file is 
rejected. After the header has been read the file is treated as an \HTK\ file.


\mysect{Waveform File Formats}{waveform}

For reading waveform data files, \HTK\ can support a variety of different
formats and these are all briefly described in this section.  The default
speech file format is \HTK. If a different format is to be used, it can be
specified by setting the configuration parameter
\texttt{SOURCEFORMAT}\index{sourceformat@\texttt{SOURCEFORMAT}}.  However,
since file formats need to be changed often, they can also be set individually
via the \texttt{-F}\index{standard options!aaaf@\texttt{-F}} command-line
option.  This over-rides any setting of the \texttt{SOURCEFORMAT} configuration
parameter.

Similarly for the output of waveforms, the format can be set using either the
configuration parameter \texttt{TARGETFORMAT} or the \texttt{-O} command-line
option.  However, for output only native \HTK\ format (\texttt{HTK}), Esignal
format (\texttt{ESIG}) and headerless (\texttt{NOHEAD}) waveform files are
supported.

The following sub-sections give a brief description of each of the waveform
file formats supported by \HTK.

\subsection{HTK File Format}

\index{file formats!HTK}
The \HTK\ file format for waveforms is identical to that described in
section~\ref{s:parmstore} above.  It consists of a 12 byte header followed
by a sequence of 2 byte integer speech samples.  For waveforms, the
\texttt{sampSize} field will be 2 and the \texttt{parmKind} field will be 0.
The \texttt{sampPeriod} field gives the sample period in 100ns units, hence for
example, it will have the value 1000 for speech files sampled at 10kHz and 625
for speech files sampled at 16kHz.

\subsection{Esignal File Format}

\index{file formats!Esignal}
The Esignal file format for waveforms is similar to that described in
section~\ref{s:parmstore} above with the following exceptions. When reading an
ESIG waveform file the \HTK\ programs only check whether the record length
equals 2 and whether the datatype of the only field in the data records is
\texttt{SHORT}. The data field that is created on output of a waveform is
called \texttt{WAVEFORM}.

\subsection{TIMIT File Format}

\index{file formats!TIMIT}
The TIMIT format has the same structure as the HTK format except that the
12-byte header contains the following

\begin{tabbing}
++ \= +++++++++ \= \kill
\>\texttt{hdrSize}\>-- number of bytes in header ie 12 (2-byte integer) \\
\>\texttt{version}\>-- version number (2-byte integer) \\
\>\texttt{numChannels}\>-- number of channels (2-byte integer) \\
\>\texttt{sampRate}\>-- sample rate (2-byte integer) \\
\>\texttt{nSamples}\>-- number of samples in file (4-byte integer) 
\end{tabbing}
TIMIT format data is used only on the prototype TIMIT CD ROM.

\subsection{NIST File Format}

\index{file formats!NIST}
The NIST file format is also referred to as the Sphere file format.
A NIST header consists of ASCII text.  It begins with a label of the
form  \texttt{NISTxx} where xx is a version code followed by the number
of bytes in the header.  The remainder of the header consists of
name value pairs of which \HTK\ decodes the following
\begin{tabbing}
++ \= +++++++++++++ \= \kill
\>\texttt{sample\_rate}  \>-- sample rate in Hz \\
\>\texttt{sample\_n\_bytes} \>-- number of bytes in each sample \\
\>\texttt{sample\_count} \>-- number of samples in file \\
\>\texttt{sample\_byte\_format} \>-- byte order \\
\>\texttt{sample\_coding} \>-- speech coding eg pcm, $\mu$law, shortpack \\
\>\texttt{channels\_interleaved} \>-- for 2 channel data only
\end{tabbing}
The current NIST Sphere data format\index{NIST Sphere data format} subsumes a
variety of internal data organisations.  HTK currently supports interleaved
$\mu$law used in Switchboard, Shortpack compression used in the original
version of WSJ0 and standard 16bit linear PCM as used in Resource Management,
TIMIT, etc.  It does not currently support the Shorten compression format as
used in WSJ1 due to licensing restrictions.  Hence, to read WSJ1, the files
must be converted using the NIST supplied decompression routines into standard
16 bit linear PCM.  This is most conveniently done under UNIX by using the
decompression program as an input filter set via the environment variable
\texttt{HWAVEFILTER}\index{hwavefilter@\texttt{HWAVEFILTER}} (see
section~\ref{s:iopipes}).
    
For interleaved $\mu$law as used in Switchboard, the default is to add the two
channels together.  The left channel only can be obtained by setting the
environment variable \texttt{STEREOMODE} to \texttt{LEFT} and the right channel
only can be obtained by setting the environment variable \texttt{STEREOMODE} to
\texttt{RIGHT}.  \index{mu law encoded files } 

\subsection{SCRIBE File Format}

\index{file formats!SCRIBE}
The SCRIBE format is a subset of the standard laid down by the European Esprit
Programme SAM Project.  SCRIBE data files are headerless and therefore consist
of just a sequence of 16 bit sample values.  \HTK\ assumes by default that the
sample rate is 20kHz.  The configuration parameter \texttt{SOURCERATE} should
be set to over-ride this.  The byte ordering assumed for SCRIBE data files is
\texttt{VAX} (little-endian).

\subsection{SDES1 File Format}

\index{file formats!Sound Designer(SDES1)}
The SDES1 format refers to the ``Sound Designer I'' format defined by
Digidesign Inc in 1985 for multimedia and general audo applications.  It is
used for storing short monoaural sound samples.  The SDES1 header is complex
(1336 bytes) since it allows for associated display window information to be
stored in it as well as providing facilities for specifying repeat loops.  The
HTK input routine for this format just picks out the following information
\begin{tabbing}
++ \= +++++++++ \= \kill
\>\texttt{headerSize}  \>-- size of header ie 1336 (2 byte integer) \\
\>(182 byte filler) \\
\>\texttt{fileSize} \>-- number of bytes of sampled data (4 byte integer)\\
\>(832 byte filler) \\
\>\texttt{sampRate} \>-- sample rate in Hz (4 byte integer) \\
\>\texttt{sampPeriod} \>-- sample period in microseconds (4 byte integer) \\ 
\>\texttt{sampSize} \>-- number of bits per sample ie 16 (2 byte integer)
\end{tabbing}

\subsection{AIFF File Format}

\index{file formats!Audio Interchange (AIFF)}
The AIFF format was defined by Apple Computer for storing monoaural and
multichannel sampled sounds.  An AIFF file consists of a number of {\it
chunks}.  A {\it Common} chunk contains the fundamental parameters of the sound
(sample rate, number of channels, etc) and a {\it Sound Data} chunk contains
sampled audio data.  \HTK\ only partially supports AIFF since some of the
information in it is stored as floating point numbers.  In particular, the
sample rate is stored in this form and to avoid portability problems, 
\HTK\ ignores the given sample rate and assumes that it is 16kHz.  If this 
default rate is incorrect, then the true sample period should be
specified by setting the \texttt{SOURCERATE} configuration parameter.
Full details of the AIFF format are available from Apple Developer
Technical Support.

\subsection{SUNAU8 File Format}

\index{file formats!Sun audio (SUNAU8)}
The SUNAU8 format defines a subset of the ``.au'' and ``.snd'' audio file
format used by Sun and NeXT.  An SUNAU8 speech data file consists of a header
followed by 8 bit $\mu$law encoded speech samples.  The header is 28 bytes and
contains the following fields, each of which is 4 bytes
\begin{tabbing}
++ \= +++++++++ \= \kill
\>\texttt{magicNumber}  \>-- magic number 0x2e736e64 \\
\>\texttt{dataLocation} \>-- offset to start of data \\
\>\texttt{dataSize} \>-- number of bytes of data \\
\>\texttt{dataFormat} \>-- data format code which is 1 for 8 bit $\mu$law \\ 
\>\texttt{sampRate} \>-- a sample rate code which is always 8012.821 Hz \\ 
\>\texttt{numChan} \>-- the number of channels \\ 
\>\texttt{info} \>-- arbitrary character string min length 4 bytes
\end{tabbing}
No default byte ordering is assumed for this format. If the data source is
known to be different to the machine being used, then the environment variable
\texttt{BYTEORDER} must be set appropriately. Note that when used on Sun Sparc
machines with 16 bit audio device the sampling rate of 8012.821Hz is not
supported and playback will be peformed at 8KHz.

\subsection{OGI File Format}

\index{file formats!OGI}
The OGI format is similar to TIMIT. The header contains the following

\begin{tabbing}
++ \= +++++++++ \= \kill
\>\texttt{hdrSize}\>-- number of bytes in header \\
\>\texttt{version}\>-- version number (2-byte integer) \\
\>\texttt{numChannels}\>-- number of channels (2-byte integer) \\
\>\texttt{sampRate}\>-- sample rate (2-byte integer) \\
\>\texttt{nSamples}\>-- number of samples in file (4-byte integer) \\
\>\texttt{lendian}\>-- used to test for byte swapping (4-byte integer) 
\end{tabbing}

\subsection{WAV File Format}{}

\index{file formats!WAV} 
The WAV file format is a subset of Microsoft's RIFF specification for the
storage of multimedia files. A RIFF file starts out with a file header followed
by a sequence of data ``chunks''. A WAV file is often just a RIFF file with a
single ``WAVE'' chunk which consists of two sub-chunks - a ``fmt'' chunk
specifying the data format and a ``data'' chunk containing the actual sample
data. The WAV file header contains the following

\begin{tabbing}
++ \= +++++++++ \= \kill
\>\texttt{'RIFF'}\>-- RIFF file identification (4 bytes) \\
\>\texttt{<length>}\>-- length field (4 bytes)\\
\>\texttt{'WAVE'}\>-- WAVE chunk identification (4 bytes) \\
\>\texttt{'fmt '}\>-- format sub-chunk identification  (4 bytes) \\
\>\texttt{flength}\>-- length of format sub-chunk (4 byte integer) \\
\>\texttt{format}\>-- format specifier (2 byte integer) \\
\>\texttt{chans}\>-- number of channels (2 byte integer) \\
\>\texttt{sampsRate}\>-- sample rate in Hz (4 byte integer) \\
\>\texttt{bpsec}\>-- bytes per second (4 byte integer) \\
\>\texttt{bpsample}\>-- bytes per sample (2 byte integer) \\
\>\texttt{bpchan}\>-- bits per channel (2 byte integer) \\
\>\texttt{'data'}\>-- data sub-chunk identification  (4 bytes) \\
\>\texttt{dlength}\>-- length of data sub-chunk (4 byte integer)
\end{tabbing}
Support is provided for 8-bit CCITT mu-law, 8-bit CCITT a-law, 8-bit PCM 
linear and 16-bit PCM linear - all in stereo or mono (use of \texttt{STEREOMODE} 
parameter as per NIST).  The default byte ordering assumed for \texttt{WAV} 
data files is \texttt{VAX} (little-endian).

\subsection{ALIEN and NOHEAD File Formats}

\index{file formats!ALIEN}
\index{file formats!NOHEAD}
\HTK\ tools can read speech waveform files with alien formats provided that
their overall structure is that of a header followed by data.  This is done by
setting the format to \texttt{ALIEN} and setting the environment variable
\texttt{HEADERSIZE} to the number of bytes in the header.  \HTK\ will then
attempt to infer the rest of the information it needs.  However, if input is
from a pipe, then the number of samples expected must be set using the
environment variable \texttt{NSAMPLES}\index{nsamples@\texttt{NSAMPLES}}.  The
sample rate of the source file is defined by the configuration parameter
\texttt{SOURCERATE} as described in section~\ref{s:sigproc}.  If the file has
no header then the format \texttt{NOHEAD} may be specified instead of
\texttt{ALIEN}\index{alien@\texttt{ALIEN}} in which case
\texttt{HEADERSIZE}\index{headersize@\texttt{HEADERSIZE}} is assumed to be
zero.

\mysect{Direct Audio Input/Output}{audioio}

\index{speech input!direct audio}
Many \HTK\ tools, particularly recognition tools, can input speech waveform
data directly from an audio device.  The basic mechanism for doing this is to
simply specify the \texttt{SOURCEKIND} as being
\texttt{HAUDIO}\index{haudio@\texttt{HAUDIO}} following which speech samples
will be read directly from the host computer's audio input device.

Note that for live audio input, the configuration variable 
\texttt{ENORMALISE} should be set to false both during training and recognition. Energy normalisation cannot
be used with live audio input, and the default setting for this variable
is \texttt{TRUE}. When training models for live audio input, be sure to
set \texttt{ENORMALISE} to false. If you have existing models trained with 
\texttt{ENORMALISE} set to true, you can retrain them using {\it single-pass
retraining} (see section~\ref{s:singlepass}).

When using direct audio input\index{direct audio input}, the input sampling
rate may be set explicitly using the configuration parameter
\texttt{SOURCERATE}, \index{sourcerate@\texttt{SOURCERATE}} otherwise 
\HTK\ will assume that it has been set by some external means such as an
audio control panel.  In the latter case, it must be possible for
\htool{HAudio} to obtain the sample rate from the audio driver
otherwise an error message will be generated.

Although the detailed control of audio hardware is typically machine dependent,
\HTK\ provides a number of Boolean configuration variables to request specific
input and output sources.  These are indicated by the following table
\begin{center}\index{audio source}\index{audio output}
\begin{tabular}{|c|l|} \hline
Variable   & Source/Sink \\ \hline
\texttt{LINEIN}   & line input \\ 
\texttt{MICIN}   & microphone input \\ 
\texttt{LINEOUT}   & line output \\
\texttt{PHONESOUT}   & headphones output \\ 
\texttt{SPEAKEROUT} & speaker output \\ \hline
\end{tabular}
\end{center}
\index{linein@\texttt{LINEIN}}
\index{micin@\texttt{MICIN}}
\index{lineout@\texttt{LINEOUT}}
\index{phonesout@\texttt{PHONESOUT}}
\index{speakerout@\texttt{SPEAKEROUT}}

The major complication in using direct audio is in starting and stopping the
input device.  The simplest approach to this is for \HTK\ tools to take direct
control and, for example, enable the audio input for a fixed period determined
via a command line option.  However, the \htool{HAudio}/\htool{HParm} modules
provides two more powerful built-in facilities for audio input control.

\index{direct audio input!silence detector!speech detector} 
The first method of audio input control involves the use of an automatic
energy-based speech/silence detector which is enabled by setting the
configuration parameter
\texttt{USESILDET}\index{usesildet@\texttt{USESILDET}} to true. Note that
the speech/silence detector can also operate on waveform input files.

The automatic speech/silence detector uses a two level algorithm which first
classifies each frame of data as either speech or silence and then applies a
heuristic to determine the start and end of each utterance.\index{HParm!SILENERGY} \index{HParm!SPEECHTHRESH}The detector classifies each
frame as speech or silence based solely on the log energy of the signal. When
the energy value exceeds a threshold the frame is marked as speech otherwise as
silence. The threshold is made up of two components both of which can be set by
configuration variables. The first component represents the mean energy level
of silence and can be set explicitly via the configuration
parameter \texttt{SILENERGY}. However, it is more usual to take a measurement
from the environment directly. Setting the configuration parameter
\texttt{MEASURESIL} to true will cause the detector to calibrate its parameters
from the current acoustic environment just prior to sampling. The second
threshold component is the level above which frames are classified as speech 
(\texttt{SPEECHTHRESH}) .
\index{HParm!SPCSEQCOUNT} \index{HParm!SPCGLCHCOUNT} \index{HParm!SILGLCHCOUNT}
Once each frame has been classified as speech or silence they are grouped into
windows consisting of \texttt{SPCSEQCOUNT} consecutive frames.  When the number
of frames marked as silence within each window falls below a glitch count the
whole window is classed as speech.  Two separate glitch counts are used, {\tt
SPCGLCHCOUNT} before speech onset is detected and {\tt SILGLCHCOUNT} whilst
searching for the end of the utterance.  This allows the algorithm to take
account of the tendancy for the end of an utterance to be somewhat quieter than
the beginning.
\index{HParm!SILMARGIN} \index{HParm!SILSEQCOUNT} 
Finally, a top level heuristic is used to determine the start and end of the
utterance. The heuristic defines the start of speech as the beginning of the
first window classified as speech.  The actual start of the processed utterance
is \texttt{SILMARGIN} frames before the detected start of speech to ensure that
when the speech detector triggers slightly late the recognition accuracy is not
affected.  Once the start of the utterance has been found the detector searches
for \texttt{SILSEQCOUNT} windows all classified as silence and sets the end of
speech to be the end of the last window classified as speech.  Once again the
processed utterance is extended \texttt{SILMARGIN} frames to ensure that if the
silence detector has triggered slightly early the whole of the speech is still
available for further processing.

\centrefig{endpointer}{120}{Endpointer Parameters}

Fig~\href{f:endpointer} shows an example of the speech/silence detection
process. The waveform data is first classified as speech or silence at frame
and then at window level before finally the start and end of the utterance are
marked. In the example, audio input starts at point {\tt A} and is stopped
automatically at point {\tt H}. The start of speech, {\tt C}, occurs when a
window of \texttt{SPCSEQCOUNT} frames are classified as speech and the start of
the utterance occurs \texttt{SILMARGIN} frames earlier at {\tt B}. The period
of silence from {\tt D} to {\tt E} is not marked as the end of the utterance
because it is shorter than \texttt{SILSEQCOUNT}. However after point {\tt F}
no more windows are classified as speech (although a few frames are) and so
this is marked as the end of speech with the end of the utterance extended to
{\tt G}.

\index{direct audio input!signal control!keypress}
The second built-in mechanism for controlling audio input is by arranging for 
a signal to be sent from some other process. Sending the signal for the first
time starts the audio device. If the speech detector is not enabled then
sampling starts immediately and is stopped by sending the signal a second
time. If automatic speech/silence detection is enabled, then the first signal 
starts the detector. Sampling stops immediately when a second signal is
received or when silence is detected. The signal number is set using the 
configuration parameter \texttt{AUDIOSIG}\index{audiosig@\texttt{AUDIOSIG}}. 
Keypress control operates in a similar fashion and is enabled by setting the 
configuration parameter \texttt{AUDIOSIG} to a negative number. In this mode
an initial keypress will be required  to start sampling/speech detection and
a second keypress will stop sampling immediately.

Audio output\index{audio output} is also supported by \HTK. There are no
generic facilities for output and the precise behaviour will depend on the tool
used. It should be noted, however, that the audio input facilities provided by
\htool{HAudio} include provision for attaching a \textit{replay buffer} to an
audio input channel.  This is typically used to store the last few seconds of
each input to a recognition tool in a circular buffer so that the last
utterance input can be replayed on demand.


\mysect{Multiple Input Streams}{streams}

\index{multiple streams}
As noted in section~\ref{s:genio}, \HTK\ tools regard the input observation
sequence as being divided into a number of independent \textit{data streams}.
For building continuous density HMM systems, this facility is of limited
use and by far the most common case is that of a single data stream.
However, when building tied-mixture systems or when
using vector quantisation, a more uniform coverage of the
acoustic space is obtained by separating energy, deltas, etc., into
separate streams.

This separation of parameter vectors into streams takes place at 
the point where the vectors
are extracted from the converted input file or audio device and
transformed into an observation.  The tools for HMM construction
and for recognition thus view the input data as a sequence of observations
but note that this is entirely internal to \HTK.   Externally data is
always stored as a single sequence of parameter vectors.

When multiple streams\index{multiple streams!rules for} are required,
the division of the parameter vectors is performed automatically
based on the parameter kind.  This works according 
to the following rules.  

\begin{description}
\item[1 stream] single parameter vector.  This is the default case.
\item[2 streams] if the parameter vector contains energy terms, then
     they are extracted and placed in stream 2.  Stream 1 contains the
     remaining static coefficients and their deltas and accelerations,
     if any.  Otherwise, 
     the parameter vector must have appended delta
     coefficients and no appended acceleration coefficients.  
     The vector is then split so that the static
     coefficients form stream 1 and
     the corresponding delta coefficients form stream 2.
\item[3 streams] 
     if the parameter vector has acceleration coefficients, then vector
     is split with static coefficients plus any energy in stream 1,
     delta coefficients plus any delta energy in stream 2 and
     acceleration coefficients plus any acceleration energy in stream 3.
     Otherwise, the parameter vector must include log energy and
     must have appended delta coefficients.  The vector is then split
     into three parts so that the static coefficients form stream
     1, the delta coefficients form stream 2, and the log energy
     and delta log energy are combined to form stream 3.  
\item[4 streams]
    the parameter vector must include log energy and
    must have appended delta and acceleration coefficients.
    The vector is split into 4 parts so that the static coefficients form stream
    1, the delta coefficients form stream 2, the acceleration
    coefficients form stream 3 and the log energy, delta energy
    and acceleration energy are combined to form stream 4.
\end{description}
In all cases, 
the static log energy can be 
suppressed (via the \texttt{\_N}\index{qualifiers!aaan@\texttt{\_N}} qualifier).
If none of the above rules apply for some required number of
streams, then the parameter vector is simply incompatible with that
form of observation.  For example, the parameter kind \texttt{LPC\_D\_A}
cannot be split into 2 streams, instead 3 streams should be used.
\index{energy suppression}

\putfig{streams}{100}{Example Stream Construction}

Fig.~\href{f:streams} illustrates the way that streams are constructed
for a number of common cases.  As earlier, the choice of \texttt{LPC}
as the static coefficients is purely for illustration and the same
mechanism applies to all base parameter kinds.

As discussed further in the next section,
multiple data streams are often used with vector quantised data.  In this
case, each VQ symbol per input sample is placed in a separate data stream.

\mysect{Vector Quantisation}{vquant}

Although \HTK\ was designed primarily for building continuous density HMM
systems, it also supports discrete density HMMs.  Discrete HMMs are
particularly useful for modelling data which is naturally symbolic. They can
also be used with continuous signals such as speech by quantising each speech
vector to give a unique VQ symbol for each input frame. The \HTK\ module
\htool{HVQ} provides a basic facility for performing this vector
quantisation\index{vector quantisation}. The VQ table (or codebook) can be
constructed using the \HTK\ tool \htool{HQuant}.

When used with speech, the principle justification for using discrete HMMs is
the much reduced computation. However, the use of vector quantisation
introduces errors and it can lead to rather fragile systems.  For this reason,
the use of continuous density systems is generally preferred.  To facilitate
the use of continuous density systems when there are computational constraints,
\HTK\ also allows VQ to be used as the basis for pre-selecting a subset of
Gaussian\index{Gaussian pre-selection} components for evaluation at each time
frame.

\sidefig{VQUse}{65}{Using Vector Quantisation}{2}{ Fig.~\href{f:VQUse}
illustrates the different ways that VQ can be used in \HTK\ for a single data
stream. For multiple streams, the same principles are applied to each stream
individually. A converted speech waveform or file of parameter vectors can
have VQ indices attached simply by specifying the name of a VQ table using the
configuration parameter \texttt{VQTABLE}\index{vqtable@\texttt{VQTABLE}} and by
adding the \texttt{\_V} qualifier to the target kind.  The effect of this is
that each \textit{observation} passed to a recogniser can include both a
conventional parameter vector and a VQ index.  \index{vector quantisation!uses
of} \index{qualifiers!aaav@\texttt{\_V}} For continuous density HMM systems, a
possible use of this might be to preselect Gaussians for evaluation (but note
that \HTK\ does not currently support this facility).

When used with a discrete HMM system, the
continuous parameter vectors are ignored and only the VQ 
indices are used.
For training and evaluating discrete HMMs, it is convenient to
store speech data in vector quantised form.  This is done using
the tool \htool{HCopy} to read in and vector quantise each speech file.
Normally, \htool{HCopy} copies the target form directly into the 
output file.  However, if the configuration parameter \texttt{SAVEASVQ}
is set, then it will
store only the VQ indices and mark the kind of the newly created
file as \texttt{DISCRETE}.  Discrete files created in this
way can be read
directly by \htool{HParm} and the VQ symbols passed directly to
a tool as indicated by the lower part of Fig.~\href{f:VQUse}.
}
\index{saveasvq@\texttt{SAVEASVQ}} 
\index{discrete@\texttt{DISCRETE}}

\index{vector quantisation!distance metrics}
\htool{HVQ} supports three types of distance metric and two organisations of
VQ codebook.   Each codebook consists of a collection of nodes where each
node has a mean vector and optionally a covariance matrix or diagonal
variance vector.  The corresponding distance metric used for  each of these
is simple Euclidean,  full covariance Mahalanobis or diagonal covariance
Mahalanobis. The codebook nodes are  arranged in 
the form of a simple linear table
or as a  binary tree. In the linear case, the input vector is compared with
every node in turn and the nearest determines the VQ index.  In the binary
tree case, each non-terminal node has a left and a right daughter.  Starting
with the top-most root node,  the input is compared with the left and right
daughter node and the nearest is selected.  This process is repeated until a
terminal node is reached. \index{vector quantisation!type of}
 

\index{vector quantisation!code book external format}\index{files!VQ codebook}
VQ Tables are stored externally in text files consisting of a header
followed by a sequence of node entries.  The header consists of the
following information
\begin{tabbing}
++ \= +++++++ \= + \= \kill
\> \textit{magic}\>  --\> a magic number usually the original parameter kind \\
\> \textit{type} \>  --\> 0 = linear tree, 1 = binary tree \\
\> \textit{mode} \>  --\> 1 = diagonal covariance Mahalanobis \\
\>\>\>  2 = full covariance Mahalanobis \\
\>\>\>  5 = Euclidean \\
\> \textit{numNodes} \> --\>  total number of nodes in the codebook \\
\> \textit{numS}\>  --\> number of independent data streams \\
\> \textit{sw1,sw2,...}\>  --\> width of each data stream \\
\end{tabbing}
Every node has a unique integer identifier and consists of the 
following
\begin{tabbing}
++ \= +++++++ \= + \= \kill
\> \textit{stream}\>  --\>stream number for this node \\
\> \textit{vqidx}\>  --\>VQ index for this node (0 if non-terminal) \\
\> \textit{nodeId}\>  --\>integer id of this node \\
\> \textit{leftId}\>  --\>integer id of left daughter node \\
\> \textit{rightId}\>  --\>integer id of right daughter node \\
\> \textit{mean}\>  --\>mean vector \\
\> \textit{cov}\>  --\>diagonal variance or full covariance \\
\end{tabbing}
The inclusion of the optional variance vector or covariance matrix depends
on the mode in the header.  If present they are stored in inverse form.
In a binary tree, the root id is always 1.  In linear codebooks, the
left and right daughter node id's are ignored.

\mysect{Viewing Speech with \htool{HList}}{UseHList}

\index{speech input!monitoring}
As mentioned in section~\ref{s:genio}, the tool \htool{HList}\index{hlist@\htool{HList}} provides
a dual r\^{o}le in \HTK.  Firstly, it can be used for examining the contents
of speech data files.  
In general, \htool{HList} displays three types of information
\begin{enumerate}
 \item \textit{source header}: requested using the \texttt{-h} option
 \item \textit{target header}: requested using the \texttt{-t} option
 \item \textit{target data}: printed by default.  The begin and end samples of the
displayed data can be specified using the \texttt{-s} and \texttt{-e} options.
\end{enumerate}
When the default configuration parameters are used, 
no conversions are applied and the target
data is identical to the contents of the file.  

\index{files!listing contents}
As an example, suppose that the file called \texttt{timit.wav} holds speech
waveform data using the TIMIT format.  The command
\begin{verbatim}
    HList -h -e 49 -F TIMIT timit.wav
\end{verbatim}
would display the source header information and the first 50 samples of the
file.  The output would look something like the following

\begin{list}{}{\setlength{\leftmargin}{-1cm}}
\item
\begin{verbatim}
         ----------------------------- Source: timit.wav ---------------------------
           Sample Bytes:  2        Sample Kind:   WAVEFORM
           Num Comps:     1        Sample Period: 62.5 us
           Num Samples:   31437    File Format:   TIMIT
         ------------------------------ Samples: 0->49 -----------------------------
             0:     8     -4     -1      0     -2     -1     -3     -2      0      0
            10:    -1      0     -1     -2     -1      1      0     -1     -2      1
            20:    -2      0      0      0      2      1     -2      2      1      0
            30:     1      0      0     -1      4      2      0     -1      4      0
            40:     2      2      1     -1     -1      1      1      2      1      1
         ------------------------------------ END ----------------------------------
\end{verbatim}
\end{list}
The source information confirms that the file contains \texttt{WAVEFORM}
data with 2 byte samples and 31437 samples in total.  The sample period is
$62.5\mu s$ which corresponds to a 16kHz sample rate.  
The displayed data is numerically small because it corresponds to leading silence.
Any part of the file could be viewed by suitable choice of the begin and end
sample indices. For example,
\begin{verbatim}
   HList -s 5000 -e 5049 -F TIMIT timit.wav
\end{verbatim}
would display samples 5000 through to 5049.  
The output might look like the following
\begin{list}{}{\setlength{\leftmargin}{-1cm}}
\item
\begin{verbatim}
         ---------------------------- Samples: 5000->5049 --------------------------
          5000:    85   -116   -159   -252     23     99     69     92     79   -166
          5010:  -100   -123   -111     48    -19     15    111     41   -126   -304
          5020:  -189     91    162    255     80   -134   -174    -55     57    155
          5030:    90     -1     33    154     68   -149    -70     91    165    240
          5040:   297     50     13     72    187    189    193    244    198    128
         ------------------------------------ END ----------------------------------
\end{verbatim}
\end{list}

The second use of \htool{HList} is to check that input conversions
are being performed properly.  Suppose that the above TIMIT format file is
part of a database to be used for training a recogniser and that mel-frequency
cepstra are to be used along with energy and the first differential coefficients.
Suitable configuration parameters needed to achieve this might be as follows
\begin{verbatim}
    # Wave -> MFCC config file
    SOURCEFORMAT = TIMIT    # same as -F TIMIT
    TARGETKIND   = MFCC_E_D # MFCC + Energy + Deltas
    TARGETRATE   = 100000   # 10ms frame rate
    WINDOWSIZE   = 200000   # 20ms window
    NUMCHANS     = 24       # num filterbank chans
    NUMCEPS      = 8        # compute c1 to c8
\end{verbatim}
\htool{HList} can be used to check this.  For example, typing
\begin{verbatim}
    HList -C config -o -h -t -s 100 -e 104 -i 9  timit.wav
\end{verbatim}
will cause the waveform file to be converted, then the source header, 
the target header and parameter vectors 100 through to 104 to be listed.
A typical output would be as follows
\begin{verbatim}
   ------------------------------ Source: timit.wav ---------------------------
     Sample Bytes:  2        Sample Kind:   WAVEFORM
     Num Comps:     1        Sample Period: 62.5 us
     Num Samples:   31437    File Format:   TIMIT
   ------------------------------------ Target --------------------------------
     Sample Bytes:  72       Sample Kind:   MFCC_E_D
     Num Comps:     18       Sample Period: 10000.0 us
     Num Samples:   195      File Format:   HTK
   -------------------------- Observation Structure ---------------------------
   x:    MFCC-1  MFCC-2  MFCC-3  MFCC-4  MFCC-5  MFCC-6  MFCC-7  MFCC-8       E
          Del-1   Del-2   Del-3   Del-4   Del-5   Del-6   Del-7   Del-8    DelE
   ------------------------------ Samples: 100->104 ---------------------------
   100:   3.573 -19.729  -1.256  -6.646  -8.293 -15.601 -23.404  10.988   0.834
          3.161  -1.913   0.573  -0.069  -4.935   2.309  -5.336   2.460   0.080
   101:   3.372 -16.278  -4.683  -3.600 -11.030  -8.481 -21.210  10.472   0.777
          0.608  -1.850  -0.903  -0.665  -2.603  -0.194  -2.331   2.180   0.069
   102:   2.823 -15.624  -5.367  -4.450 -12.045 -15.939 -22.082  14.794   0.830
         -0.051   0.633  -0.881  -0.067  -1.281  -0.410   1.312   1.021   0.005
   103:   3.752 -17.135  -5.656  -6.114 -12.336 -15.115 -17.091  11.640   0.825
         -0.002  -0.204   0.015  -0.525  -1.237  -1.039   1.515   1.007   0.015
   104:   3.127 -16.135  -5.176  -5.727 -14.044 -14.333 -18.905  15.506   0.833
         -0.034  -0.247   0.103  -0.223  -1.575   0.513   1.507   0.754   0.006
   ------------------------------------- END ----------------------------------
\end{verbatim}

The target header information shows that the converted data consists
of 195 parameter vectors, each vector having 18 components and being 72 bytes in
size.  The structure of each parameter vector is displayed as a simple sequence
of floating-point numbers.   The layout information described in 
section~\ref{s:parmstore} can be used to interpret the data.  
However, including the \texttt{-o} option, as in the example, causes \htool{HList}
to output a schematic of the observation structure.  Thus, it can be seen that
the first row of each sample contains the static coefficients and the second 
contains the delta
coefficients.  The energy is in the final column.
The command line option \texttt{-i 9} controls the number of values displayed
per line and can be used to aid in the visual interpretation of the data.
Notice finally that the command line option \texttt{-F TIMIT} was not required
in this case because the source format was specified in the configuration file.

It should be stressed that when \htool{HList} displays parameterised data,
it does so in exactly the form that \textit{observations} are passed to
a \HTK\ tool.  So, for example, if the above data was input to a system
built using 3 data streams, then this can be simulated by using the
command line option \texttt{-n} to set the number of streams.  For example, typing
\begin{verbatim}
    HList -C config -n 3 -o -s 100 -e 101 -i 9  timit.wav
\end{verbatim}
would result in the following output
\begin{verbatim}
      ------------------------ Observation Structure -----------------------
      nTotal=18 nStatic=8 nDel=16  eSep=T
      x.1:    MFCC-1  MFCC-2  MFCC-3  MFCC-4  MFCC-5  MFCC-6  MFCC-7  MFCC-8
      x.2:     Del-1   Del-2   Del-3   Del-4   Del-5   Del-6   Del-7   Del-8
      x.3:         E    DelE
      -------------------------- Samples: 100->101 -------------------------
      100.1:   3.573 -19.729  -1.256  -6.646  -8.293 -15.601 -23.404  10.988
      100.2:   3.161  -1.913   0.573  -0.069  -4.935   2.309  -5.336   2.460
      100.3:   0.834   0.080
      101.1:   3.372 -16.278  -4.683  -3.600 -11.030  -8.481 -21.210  10.472
      101.2:   0.608  -1.850  -0.903  -0.665  -2.603  -0.194  -2.331   2.180
      101.3:   0.777   0.069
      --------------------------------- END --------------------------------
\end{verbatim}
Notice that the data is identical to the previous case, but it has been
re-organised into separate streams.\index{observations!displaying structure of}

\mysect{Copying and Coding using \htool{HCopy}}{UseHCopy}

\index{files!copying}
\htool{HCopy}\index{hcopy@\htool{HCopy}} is a general-purpose tool 
for copying and manipulating speech files.
The general form of invocation is
\begin{verbatim}
    HCopy src tgt
\end{verbatim}
which will make a new copy called \texttt{tgt} of the file called \texttt{src}.
\htool{HCopy} can also concatenate several sources together as in 
\begin{verbatim}
    HCopy src1 + src2 + src3 tgt    
\end{verbatim}
which concatenates the contents of \texttt{src1}, \texttt{src2} and \texttt{src3},
storing the results in the file \texttt{tgt}.  As well as putting speech files
together, \htool{HCopy} can also take them apart.  For example,
\begin{verbatim}
    HCopy -s 100 -e -100 src tgt
\end{verbatim}
will extract samples 100 through to N-100 of the file \texttt{src} to the file
\texttt{tgt} where N is the total number of samples in the source file.
The range of samples to be copied can also be specified with reference to
a label file, and modifications made to the speech file can be tracked in a 
copy of the label file.  All of the various options provided by \htool{HCopy} 
are given in the reference section and in total they provide a powerful
facility for manipulating speech data files.

However, the use of \htool{HCopy} extends beyond that of copying, chopping
and concatenating files.  \htool{HCopy} reads in all files using the speech 
input/output subsystem described in the preceding sections.  Hence, by specifying
an appropriate configuration file, \htool{HCopy} is also a speech coding tool.
For example, if the configuration file \texttt{config} was set-up to convert
waveform data to MFCC coefficients, the command
\begin{verbatim}
    HCopy -C config -s 100 -e -100 src.wav tgt.mfc
\end{verbatim}
would parameterise the file waveform file \texttt{src.wav}, excluding the
first and last 100 samples, and store the result in \texttt{tgt.mfc}.

\htool{HCopy} will process its arguments in pairs, and as with all \HTK\ tools,
argument lists can be written in a script file specified via the \texttt{-S}
option.  When coding a large database, the separate invocation of \htool{HCopy} 
for each file needing to be processed would incur a very large overhead.  Hence,
it is better to create a file, \texttt{flist} say, containing a list of
all source and target files, as in for example,
\begin{verbatim}
    src1.wav tgt1.mfc
    src2.wav tgt2.mfc
    src3.wav tgt3.mfc
    src4.wav tgt4.mfc
    etc
\end{verbatim}
and then invoke \htool{HCopy} by
\begin{verbatim}
    HCopy -C config -s 100 -e -100 -S flist
\end{verbatim}
which would encode each file listed in \texttt{flist} in a single invocation.

Normally \htool{HCopy} makes a direct copy of the target speech data in the
output file.  However, if the configuration 
parameter \texttt{SAVECOMPRESSED}\index{savecompressed@\texttt{SAVECOMPRESSED}}
is set true then the output is saved in compressed 
form and if the 
configuration parameter \texttt{SAVEWITHCRC}\index{savewithcrc@\texttt{SAVEWITHCRC}}
is set true then a checksum is appended to the output 
(see section~\ref{s:parmstore}).  If the configuration 
parameter \texttt{SAVEASVQ} is set true then only
VQ indices are saved and the kind of the target file is changed to
\texttt{DISCRETE}.  For this to work, the target kind must have the
qualifier \texttt{\_V}
\index{qualifiers!aaav@\texttt{\_V}}
attached (see section~\ref{s:vquant}). 
\index{compression}\index{check sums}
\index{files!compressing}\index{files!adding checksums}

\centrefig{coercions}{100}{Valid Parameter Kind Conversions}

\mysect{Version 1.5 Compatibility}{v1spcompat}

The redesign of the \HTK\ front-end in version 2 has introduced
a number of differences in parameter encoding.  The main
changes are
\begin{enumerate}
   \item  Source waveform zero mean processing is now performed on a frame-by-frame
   basis.
   \item Delta coefficients use a modified form of regression rather than
   simple differences at the start and end of the utterance.
   \item Energy scaling is no longer applied to the zero'th MFCC coefficient.
\end{enumerate}
If a parameter encoding is required which is as close as possible
to the version 1.5 encoding, then the compatibility configuration
variable \texttt{V1COMPAT} should be set to true.

Note also in this context that the default values for the various
configuration values have been chosen to be consistent with the
defaults or recommended practice for version 1.5.

\mysect{Summary}{spiosum}

\index{speech input!summary of variables}
This section summarises the various file formats, parameter kinds, qualifiers
and configuration parameters used by \HTK.  Table~\href{t:fileform} lists the
audio speech file formats which can be read by the \htool{HWave} module.
Table~\href{t:parmkinds} lists the basic parameter kinds supported by the
\htool{HParm} module and Fig.~\href{f:coercions} shows the various automatic
conversions that can be performed by appropriate choice of source and target
parameter kinds.  Table~\href{t:qualifiers} lists the available qualifiers for
parameter kinds.  The first 6 of these are used to describe the target kind.
The source kind may already have some of these, \htool{HParm} adds the rest as
needed.  Note that \htool{HParm} can also delete qualifiers when converting
from source to target.  The final two qualifiers in Table~\href{t:qualifiers}
are only used in external files to indicate compression and an attached
checksum.  \htool{HParm} adds these qualifiers to the target form during output
and only in response to setting the configuration parameters
\texttt{SAVECOMPRESSED} and \texttt{SAVEWITHCRC}.  Adding the
\texttt{\_C}\index{qualifiers!aaac@\texttt{\_C}} or
\texttt{\_K}\index{qualifiers!aaak@\texttt{\_K}} qualifiers to the target kind
simply causes an error. Finally, Tables \href{t:spiocparms1} and
\href{t:spiocparms2} lists all of the configuration parameters along with their
meaning and default values.

\begin{center}
\begin{tabular}{|p{2.6cm}|p{8.7cm}|} \hline
Name &  Description  \\ \hline
\texttt{\HTK} &  The standard \HTK\ file format\\
\texttt{TIMIT} &  As used in the original prototype TIMIT CD-ROM\\
\texttt{NIST} &  The standard SPHERE format used by the US NIST\\
\texttt{SCRIBE} &  Subset of the European SAM standard used in the SCRIBE CD-ROM\\
\texttt{SDES1} &  The Sound Designer 1 format defined by Digidesign Inc. \\
\texttt{AIFF} &  Audio interchange file format\\
\texttt{SUNAU8} &  Subset of 8bit ".au" and ".snd" formats used by Sun and NeXT\\
\texttt{OGI} &  Format used by Oregan Graduate Institute similar to TIMIT\\
\texttt{WAV} & Microsoft WAVE files used on PCs\\
\texttt{ESIG} &  Entropic Esignal file format\\ \hline
\texttt{AUDIO} & Pseudo format to indicate direct audio input \\
\texttt{ALIEN} & Pseudo format to indicate unsupported file,  the 
       alien header size must be set via the environment variable \texttt{HDSIZE} \\
\texttt{NOHEAD} & As for the ALIEN format but header size is zero \\ \hline
\end{tabular}
\tabcap{fileform}{Supported File Formats}
\end{center}

\begin{center}
\begin{tabular}{|p{2.6cm}|p{8.7cm}|} \hline
Kind &  Meaning  \\ \hline
\texttt{WAVEFORM} & scalar samples (usually raw speech data)  \\
\texttt{LPC} & linear prediction coefficients  \\
\texttt{LPREFC} & linear prediction reflection coefficients  \\
\texttt{LPCEPSTRA}   & LP derived cepstral coefficients  \\
\texttt{LPDELCEP} & LP cepstra + delta coef (obsolete)  \\
\texttt{IREFC} & LPREFC stored as 16bit (short) integers  \\
\texttt{MFCC}  & mel-frequency cepstral coefficients  \\
\texttt{FBANK} & log filter-bank parameters  \\
\texttt{MELSPEC} & linear filter-bank parameters  \\
\texttt{USER} & user defined parameters  \\
\texttt{DISCRETE} &  vector quantised codebook symbols  \\
\texttt{PLP} &  perceptual linaer prediction coefficients  \\
\texttt{ANON} & matches actual parameter kind \\ \hline
\end{tabular}
\tabcap{parmkinds}{Supported Parameter Kinds}
\end{center}

\begin{center}
\begin{tabular}{|p{2.6cm}|p{8.7cm}|} \hline
Qualifier &  Meaning  \\ \hline
\texttt{\_A} &  Acceleration coefficients appended \\
\texttt{\_C} &  External form is compressed\\
\texttt{\_D} &  Delta coefficients appended \\
\texttt{\_E} &  Log energy appended\\
\texttt{\_K} &  External form has checksum appended\\
\texttt{\_N} &  Absolute log energy suppressed \\
\texttt{\_T} &  Third differential coefficients appended \\
\texttt{\_V} &  VQ index appended\\
\texttt{\_Z} &  Cepstral mean subtracted\\
\texttt{\_0} &  Cepstral C0 coefficient appended\\ \hline
\end{tabular}
\tabcap{qualifiers}{Parameter Kind Qualifiers}
\end{center}\index{qualifiers!summary}

\begin{center}
\begin{tabular}{|p{1.2cm}|p{3.0cm}|p{1.3cm}|p{6.5cm}|} \hline
Module & Name & Default & Description  \\ \hline
\htool{HAudio} & \texttt{LINEIN} & \texttt{T} & Select line input for audio\\ 
\htool{HAudio} & \texttt{MICIN} & \texttt{F} & Select microphone input for audio\\ 
\htool{HAudio} & \texttt{LINEOUT} & \texttt{T} & Select line output for audio\\ 
\htool{HAudio} & \texttt{SPEAKEROUT} & \texttt{F} & Select speaker output for audio\\ 
\htool{HAudio} & \texttt{PHONESOUT} & \texttt{T} & Select headphones output for audio\\ 

 & \texttt{SOURCEKIND} & \texttt{ANON} & Parameter kind of source \\
 & \texttt{SOURCEFORMAT} & \texttt{HTK} & File format of source \\
 & \texttt{SOURCERATE} & \texttt{0.0} & Sample period of source in 100ns units \\
\htool{HWave} & \texttt{NSAMPLES} &  & Num samples in alien file input via a pipe\\ 
\htool{HWave} & \texttt{HEADERSIZE}  &   & Size of header in an alien file\\ 
\htool{HWave} & \texttt{STEREOMODE} &   & Select channel: \texttt{RIGHT} or \texttt{LEFT} \\
\htool{HWave} & \texttt{BYTEORDER} &   & Define byte order \texttt{VAX} or other\\
 & \texttt{NATURALREADORDER}  & \texttt{F} & Enable natural read order for HTK files \\
 & \texttt{NATURALWRITEORDER} & \texttt{F} & Enable natural write order for HTK files \\
 & \texttt{TARGETKIND} & \texttt{ANON} & Parameter kind of target \\
 & \texttt{TARGETFORMAT} & \texttt{HTK} & File format of target \\
 & \texttt{TARGETRATE} & \texttt{0.0} & Sample period of target in 100ns units \\
\htool{HParm} & \texttt{SAVECOMPRESSED} & \texttt{F} & Save the output file in compressed form \\
\htool{HParm} & \texttt{SAVEWITHCRC} & \texttt{T} & Attach a checksum to output parameter file \\ 
\htool{HParm} & \texttt{ADDDITHER} & \texttt{0.0} & Level of noise added to input signal \\
\htool{HParm} & \texttt{ZMEANSOURCE} & \texttt{F} & Zero mean source waveform before analysis \\
\htool{HParm} & \texttt{WINDOWSIZE} & \texttt{256000.0} & Analysis window size in 100ns units \\
\htool{HParm} & \texttt{USEHAMMING} & \texttt{T} & Use a Hamming window \\
\htool{HParm} & \texttt{PREEMCOEF} & \texttt{0.97} & Set pre-emphasis coefficient \\
\htool{HParm} & \texttt{LPCORDER} &  \texttt{12} &  Order of LPC analysis \\
\htool{HParm} & \texttt{NUMCHANS} & \texttt{20} & Number of filterbank channels \\
\htool{HParm} & \texttt{LOFREQ} & \texttt{-1.0} & Low frequency cut-off in fbank analysis \\
\htool{HParm} & \texttt{HIFREQ} & \texttt{-1.0} & High frequency cut-off in fbank analysis \\
\htool{HParm} & \texttt{USEPOWER} & \texttt{F} & Use power not magnitude in fbank analysis \\
\htool{HParm} & \texttt{NUMCEPS} &   \texttt{12} & Number of cepstral parameters \\ 
\htool{HParm} & \texttt{CEPLIFTER} &   \texttt{22} & Cepstral liftering coefficient \\ 
\htool{HParm} & \texttt{ENORMALISE} & \texttt{T} & Normalise log energy \\ 
\htool{HParm} & \texttt{ESCALE} & \texttt{0.1} & Scale log energy \\ 
\htool{HParm} & \texttt{SILFLOOR} & \texttt{50.0} & Energy silence floor (dB) \\
\htool{HParm} & \texttt{DELTAWINDOW} & \texttt{2} & Delta window size \\ 
\htool{HParm} & \texttt{ACCWINDOW} & \texttt{2} & Acceleration window size \\ 
\htool{HParm} & \texttt{VQTABLE} & \texttt{NULL} & Name of VQ table \\ 
\htool{HParm} & \texttt{SAVEASVQ} & \texttt{F} & Save only the VQ indices \\
\htool{HParm} & \texttt{AUDIOSIG} & \texttt{0} & Audio signal number for remote control \\ \hline
\end{tabular}
\tabcap{spiocparms1}{Configuration Parameters}
\end{center}

\begin{center}
\begin{tabular}{|p{1.1cm}|p{2.6cm}|p{1.4cm}|p{6.5cm}|} \hline
Module & Name & Default & Description  \\ \hline
\htool{HParm} & \texttt{USESILDET}  & \texttt{F} & Enable speech/silence detector \\
\htool{HParm} & \texttt{MEASURESIL} & \texttt{T} & Measure background noise level prior to sampling \\
\htool{HParm} & \texttt{OUTSILWARN} & \texttt{T} & Print a warning message to {\tt stdout} before 
 measuring audio levels \\
\htool{HParm} & \texttt{SPEECHTHRESH} & \texttt{9.0} & Threshold for speech above silence level (dB) \\
\htool{HParm} & \texttt{SILENERGY}    & \texttt{0.0} & Average background noise level (dB) \\
\htool{HParm} & \texttt{SPCSEQCOUNT}  & \texttt{10}  & Window over which speech/silence decision reached \\
\htool{HParm} & \texttt{SPCGLCHCOUNT} & \texttt{0}   & Maximum number of frames marked as silence in window which is
classified as speech whilst expecting start of speech \\
\htool{HParm} & \texttt{SILSEQCOUNT}  & \texttt{100} & Number of frames classified as silence needed to mark end of
utterance \\
\htool{HParm} & \texttt{SILGLCHCOUNT} & \texttt{2} & Maximum number of frames marked as silence in window which is
classified as speech whilst expecting silence \\
\htool{HParm} & \texttt{SILMARGIN} & \texttt{40} & Number of extra frames included before and after start and end of
speech marks from the speech/silence detector \\
\htool{HParm} & \texttt{V1COMPAT} & \texttt{F} & Set Version 1.5 compatibility mode \\ 
 & \texttt{TRACE} & \texttt{0} & Trace setting\\ \hline
\end{tabular}
\tabcap{spiocparms2}{Configuration Parameters (cont)}
\end{center}


%%% Local Variables: 
%%% mode: latex
%%% TeX-master: "htkbook"
%%% End: 
