% Abstract in Chinese
\renewcommand{\thesisabstracthead}{摘要}
\renewcommand{\thesistitle}{\thesistitlechinese}
\begin{thesisabstract}
\\
本論文主要探討能量特徵重刻技術對雜訊性語音辨識的影響。
語音辨識系統常會受到環境雜訊的影響而導致辨識效能低落,
使得語音強健性技術長久以來被視為一個非常重要的研究課題。
然而過去有不少研究指出語音能量特徵對於雜訊環境下的語音辨識影響甚鉅,
因此我們提出資料驅動能量特徵重刻法~(Data-driven energy features rescaling, DEFR)~對能量特徵作進一步的調整。
此方法分為語音活動偵測、分段對數尺度函數以及參數搜尋法三個部分。
目的是希望能夠減少雜訊與乾淨語音特徵值的差異性。
我們將此方法應用在梅爾倒頻譜參數與~Teager~能量倒頻譜參數上,
並且和均值消去法與均值正規化法作比較。
我們採用~Aurora~2.0~與~Aurora~3.0~語料庫來驗證此方法之成效,
由實驗結果證實本論文所提出之方法,能夠有效地提升辨識率。

  \vspace{\baselineskip}
  \noindent
  \textbf{關鍵詞:} \Keywords
  \end{thesisabstract}

